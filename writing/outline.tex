\documentclass[onecolumn]{paper}
\usepackage{tabularx}
\newcolumntype{L}{>{\raggedright\arraybackslash}X}

\title{ConFrm: Confidentiality Framework for Crash Safe Storage Systems}
\author{Atalay Mert Ileri \and Frans Kaashoek \and Nickolai Zeldovich}
\begin{document}
\maketitle

\section{Abstract}

\section{Introduction}

\section{Confidentiality}
	\subsection{Description and Definitions}

\section{Problems}
	\subsection{Leakage via hashing}
	\paragraph*{Malicious Example}
	\begin{verbatim}
		hash <- hash_block b
		if (hash =/= 0) then
		write given_addr b
		else
		write malicious_addr b
	\end{verbatim}

	\paragraph*{Okay example}
	\begin{verbatim}
		hash <- hash_block b
		if (hash =/= 0) then
		write given_addr b
	\end{verbatim}

	A primitive can be used maliciously.\\
	There isn’t a way to limit how it’s used.\\
	Language level safety techniques are not usable if potentially exploitable primitives are in the language.

	\subsection{Leakage via nondeterminism frequencies}	
	Standard nondeterministic noninterference definition allows leakage via relative frequency of outcomes.
	\paragraph{Nondeterministic Noninterference:}
	
	\begin{verbatim}
		forall p s1 s2 s1’ v,
		R s1 s2 ->
		exec s1 p s1’ v ->
		
		exists s2’,
		exec s2 p s2’ v /\
		R s1’ s2’.
	\end{verbatim} 
	
	
	
	\paragraph*{Leaky example}
	\begin{verbatim}
		should_leak <- get_random_bit()
		if (should_leak = 1)
		return secret_bit
		else
		return get_random_bit()
	\end{verbatim}
	
	This code leaks the secret bit 50\% of the time and outputs a random bit 50\% of the time.
	Satisfies above definition but the secret bit can be determined via the observed frequency.
	
	\begin{tabular}{| c | c | c |}
		\hline
		Secret bit & Output 0 \% & Output 1 \% \\
		\hline
		0 &	75\% & 25\% \\
		\hline
		1 &	25\% & 75\% \\
		\hline
		\end{tabular}
	
	\subsection{Modularity}
\newpage
\section{ConFrm Framework}
	\subsection{Addressing hash leakage}
	Need to have a way to take advantage of the fact that they are used safely.
	Our solution is abstracting the leaky primitive away.
	Abstract language won’t have a leaky primitive anymore so language level techniques can be used.
	
		\subsubsection{Layer system}
		\subsubsection{Abstraction and Bisimulation}

	\subsection{Addressing frequency leakage}
	Our suggested solution, parameterize the execution with a nondeterminism oracle, so that execution is deterministic w.r.t. an oracle.
		
		\subsubsection{Externalizing the nondeterminism via oracles}
			%%Add an example here to show the difference between old and new semantics
			\paragraph{Deterministic w.r.t. an oracle:}
			\begin{verbatim}
				forall oracle p s s1 v1 s2 v2,
				exec oracle s p s1 v1 ->
				exec oracle s p s2 v2 ->
				s1  = s2 /\ v1 = v2.
			\end{verbatim}
	
			\paragraph{Relativized Noninterference:}
			\begin{verbatim}
				forall oracle p s1 s2 s1’ v,
				R s1 s2 ->
				exec oracle s1 p s1’ v ->
				
				exists s2’,
				exec oracle s2 p s2’ v /\
				R s1’ s2’.
			\end{verbatim}
						
			The malicious example doesn’t satisfy relativized noninterference. For the oracle [should\_leak = 1], there is no corresponding execution from the opposite secret bit value.
		\subsubsection{Generalizing simulations to oracles}

	\subsection{Compositionality}
		\subsubsection{Layer system}
		\subsubsection{Noninterference transfer}
		\subsubsection{Bisimulation to simulation via determinism}

	\subsection{Crash Support}
		\subsubsection{Crash semantics} 
		\subsubsection{Reboot functions}
		\subsubsection{Crash refinement}
\newpage
\section{Framework in Action: ConFS}
	\subsection{Design and Specs}
		\subsubsection{Layers}
			\begin{tabularx}{\linewidth}{|L|L|L|L|}
				\hline
				Layer Name &
				Operations &
				State &
				Type \\
				\hline
				Disk &
				Read, Write, Sync &
				mem A (V * list V) &
				Basic \\
				\hline
				Cache &
				Read, Write, Flush &
				mem A V &
				Basic \\
				\hline
				Crypto &
				Hash, GetKey, Encrypt, Decrypt &
				list key, mem V (key, V), mem hash (hash, V) &
				Basic \\
				\hline
				Crypto Disk &
				Crypto ops + Disk ops &
				Crypto state + Disk state &
				Composed \\
				\hline
				Cached Disk &
				Cache ops + Crypto Disk ops &
				Cache state + Crypto Disk state &
				Composed \\
				\hline
				Logged Disk &
				Read, Write, Recover &
				mem A V &
				Abstraction \\ 
				\hline
				Storage &
				Get, Put, Delete &
				option V &
				Basic \\
				\hline
				Transaction Cache &
				Storage ops + Logged Disk ops &
				Storage state + Logged Disk state &
				Composed \\
				\hline
				Transactional Disk &
				Start, Read, Write, Commit, Abort, Recover &
				mem A V , mem A V &
				Abstraction \\
				\hline
				Authentication &
				Auth &
				user &
				Basic \\
				\hline
				Authenticated Disk &
				Authentication ops + Transactional Disk ops &
				Authentication state + Transactional Disk state &
				Composed \\
				\hline
				File Disk &
				Read, Write, Extend, Set Owner, Create, Delete, Recover &
				mem A File &
				Abstraction \\
				\hline
			\end{tabularx}
	\subsection{Implementation}
		\subsubsection{Modules}
		\subsubsection{Hash collisions and encrypted log}
\newpage
\section{Evaluation}
	\subsection{Did we achieve the goal?}
	\subsection{How much work did it take?}
	\subsection{What is the performance of the file system?}

\section{Related work}

\section{Conclusion}

\end{document}