\chapter{DiskSec}

\section{Specification: data noninterference}
\label{s:spec}

{\color{red} For some reason I am not happy with this part. I may rewrite this. }\\
%%%% Rewrite this to explain it is a form of relational noninterference.
To capture the notion of confidentiality in a file system, \sys
defines the notion of \emph{data noninterference}.  Loosely speaking,
data noninterference states that two executions are indistinguishable
with respect to specific confidential data (e.g., the contents of
a file).  Data noninterference allows an application to conclude that
an adversary cannot learn the contents of a file from the file system
but may be able to learn other information about the file (e.g., its
length, its creation time, the fact that it was created at all, etc.).
Furthermore, data noninterference does not place any restrictions on
application code, which captures the discretionary aspect of typical
file-system permissions.  This notion intuitively corresponds to the
security guarantees provided by Linux file systems.

%\begin{figure}[ht]
%  \centering
%    \begin{tikzpicture}[>=latex]

  \tikzstyle{state}=[circle, draw, minimum size=.8cm, node distance=2.5cm];
  \tikzstyle{ret}=[];
  \tikzstyle{exec}=[rectangle, rounded corners=3mm, opacity=0.2, fill=black, minimum width=8cm, minimum height=1.5cm];

  \tikzstyle{sim}=[<->, decorate, decoration={snake, pre length=2mm, post length=2mm, segment length=2mm}];
  \tikzstyle{syscall}=[->];
  \tikzstyle{sysret}=[->];
  \tikzstyle{retsim}=[<->, densely dotted, thick];

  \draw node (s0)  [state]              {$s_0$};
  \draw node (s'0) [state, below of=s0] {$s'_0$};

  \draw node (s1)  [state, right of=s0, xshift=1cm] {$s_1$};
  \draw node (s'1) [state, below of=s1] {$s'_1$};

  \draw node (s2)  [state, right of=s1, xshift=1cm] {$s_2$};
  \draw node (s'2) [state, below of=s2] {$s'_2$};

  \draw [sim] (s0) -- (s'0);
  \draw [sim] (s1) -- (s'1);
  \draw [sim] (s2) -- (s'2) {} node [pos=0.35, right] {$equivalent_{\mathrm{adv}}$};

  \draw [syscall] (s0) -- (s1)   node (s01)  [near start, below] {} node [midway, below] {$p_\mathrm{user}$};
  \draw [syscall] (s'0) -- (s'1) node (s'01) [near start, below] {} node [midway, below] {$p_\mathrm{user}$};

  \draw [ret] node (s01r)  [above of=s01,  xshift=1cm, yshift=-.2cm] {$r_0$};
  \draw [ret] node (s'01r) [above of=s'01, xshift=1cm, yshift=-.2cm] {$r'_0$};

  \draw [sysret] (s01.north)  to [out=0, in=225] (s01r);
  \draw [sysret] (s'01.north) to [out=0, in=225] (s'01r);

  \draw [syscall] (s1)  -- (s2)  node (s12)  [near start, below] {} node [midway, below] {$p_\mathrm{adv}$};
  \draw [syscall] (s'1) -- (s'2) node (s'12) [near start, below] {} node [midway, below] {$p_\mathrm{adv}$};
 
  \draw [ret] node (s12r)  [above of=s12,  xshift=1cm, yshift=-.2cm] {$r_1$};
  \draw [ret] node (s'12r) [above of=s'12, xshift=1cm, yshift=-.2cm] {$r'_1$};

  \draw [sysret] (s12.north)  to [out=0, in=225] (s12r);
  \draw [sysret] (s'12.north) to [out=0, in=225] (s'12r);
  \draw [retsim] (s12r) to [kinky cross=(s1)--(s2), kinky crosses=left] (s'12r);

  \draw node [exec, yshift=.25cm] at (s1.center) {};
  \draw node [exec, yshift=.25cm] at (s'1.center) {};

  %\draw [] node (sleg) [right of=s2] {$\cdot$};
  %\draw [] node (sleg') [right of=sleg, xshift=3mm] {$\cdot$};
  %\draw [syscall] (sleg) -- (sleg') {} node [midway, below] {$\mathrm{exec}$};

  %\draw [] node (rleg) [right of=s2, yshift=8mm] {};
  %\draw [] node (rleg') [below of=rleg, yshift=-3mm] {};
  %\draw [retsim] (rleg) -- (rleg') {} node [midway, right] {$=$};

  %\draw [] node (sleg2) [right of=s'2, yshift=8mm] {$\cdot$};
  %\draw [] node (sleg2') [below of=sleg2, yshift=-3mm] {$\cdot$};
  %\draw [sim] (sleg2) -- (sleg2') {} node [midway, right] {$equivalent_{\mathrm{adv}}$};

\end{tikzpicture}

%  \caption{Overview of \sys's approach to reasoning about confidentiality.}
%  \label{fig:approach}
%\end{figure}

\paragraph{Two-safety formulation.}
\sys formulates data noninterference in terms of two-safety, as shown
in \autoref{fig:approach}.  Specifically, data noninterference considers
two executions that run the same code but start from different states.  In
\autoref{fig:approach}, the executions are shown as horizontal transitions
between states, indicated by the gray outlines.  The executions
consist of a step by the user (running procedure $p_\mathrm{user}$,
corresponding to some system call) and then a step by the adversary
(running $p_\mathrm{adv}$, corresponding to some other system call).
Although \autoref{fig:approach} shows one particular pair of executions,
\sys's theorems consider all possible such pairs of executions.

The starting states in these two executions ($s_0$ and $s'_0$)
agree on all data visible to the adversary but could have
different contents of confidential files.  We call these two states
\emph{equivalent}$_\mathrm{adv}$, to indicate that they are equivalent
with respect to the adversary.  This equivalence is indicated by the
squiggly line in \autoref{fig:approach}.  The essence of data noninterference
is allowing the states to differ in the contents of confidential data
while requiring all other metadata (such as file length, directory order,
etc.) to remain the same.

{\color{red} Rewrite Ends Here. }\\
%%%%%%%%%%%%% Rewrite Ends Here %%%%%%%%%%%

The definition of data noninterference consists of two
requirements.  The first is \emph{state noninterference},
which requires that after every transition, the resulting states
remain \emph{equivalent}$_\mathrm{adv}$.  This is indicated in
\autoref{fig:approach} by the squiggly lines between $s_1$ and $s'_1$,
as well as between $s_2$ and $s'_2$.  This requirement ensures that
confidential data from $s_0$ and $s'_0$ does not suddenly become
accessible to the adversary in a subsequent state, and it addresses the
indirect-data-disclosure challenge.

The second requirement is \emph{return-value noninterference},
which requires that transitions by the adversary return exactly the
same values in both executions.  For example, \autoref{fig:approach}
shows that the adversary's $p_\mathrm{adv}$ returns $r_1$ in the top
execution and $r'_1$ in the bottom execution.  Return-value noninterference
requires that $r_1=r'_1$, as indicated by the dotted
arrow.  This prevents the adversary
from learning any confidential data, such as through collusion with an
adversarial file system.


\paragraph{Capturing file-system security.}

Achieving the two requirements from data noninterference ensures
that the adversary cannot obtain confidential data from the
file system.  This is because state noninterference maintains
\emph{equivalence}$_\mathrm{adv}$ regardless of what the adversary does, and any attempts by the adversary to observe
information will produce identical results, based on return-value
noninterference,
because they run in \emph{equivalent}$_\mathrm{adv}$ states.

The discretionary nature of data noninterference shows up in the fact
that legitimate users can obtain different results depending on the
confidential data.  For example, in \autoref{fig:approach}, the results
of the user's execution of $p_\mathrm{user}$, $r_0$ and $r'_0$, might be
different, because $p_\mathrm{user}$ could correspond to the user reading
a confidential file.  At this point, a user has the discretion to disclose this information
(e.g., by writing it to a public file).  Data noninterference does not
prevent this, by design, because it is attempting to model the standard
discretionary access control in a POSIX file system.


\paragraph{Defining return-value noninterference.}

\autoref{fig:two-safety-ret} presents \sys's definition of return-value
noninterference, in a simplified notation.  This definition relies on
the definition of \cc{exec}, which describes how procedures execute.
\cc{exec} takes four arguments: the procedure that is executing (\cc{p}),
the principal on whose behalf \cc{p} is running (\cc{u}), and the starting
state (\cc{st0}).
\cc{exec} returns two things: the outcome and an \emph{unseal trace},
which we describe later.
The outcome can be either \cc{Finished st' r}, indicating that the
procedure ended in state \cc{st'} and returned \cc{r}, or
\cc{Crashed st'}, indicating that the system crashed in state \cc{st'}.
The unseal traces are irrelevant for now and are used only as part of the
proof technique described in \autoref{s:proof}.
This definition also relies on a notion of two states being equivalent for
a particular principal, \cc{equivalent\_for\_principal}, which captures
the intuitive notion \emph{equivalent}$_\mathrm{adv}$ from above.

\begin{figure}[ht]
  \begin{BVerbatim}[commandchars=\\\{\},codes={\catcode`\$=3\catcode`\^=7\catcode`\_=8},fontsize=\footnotesize]
\PY{k+kn}{Definition} \PY{n}{equivalent\PYZus{}for\PYZus{}principal} \PY{n}{u} \PY{n}{st0} \PY{n}{st1} \PY{o}{:=}
  \PY{c}{(*}\PY{c}{ all parts of st0 and st1 that are accessible to}
\\[0.7\baselineskip]\PY{c}{     principal u are identical }\PY{c}{*)}\PY{o}{.}
\PY{k+kn}{Definition} \PY{n}{ret\PYZus{}noninterference} \PY{o}{`}\PY{o}{(}\PY{n}{p} \PY{o}{:} \PY{n}{proc} \PY{n}{T}\PY{o}{)} \PY{o}{:=}
  \PY{k}{forall} \PY{n}{u} \PY{n}{st0} \PY{n}{st0\PYZsq{}} \PY{n}{rand} \PY{n}{ret} \PY{n}{tr0} \PY{n}{st1}\PY{o}{,}
    \PY{n}{exec} \PY{n}{p} \PY{n}{u} \PY{n}{st0} \PY{n}{rand} \PY{o}{=}
        \PY{n}{Some} \PY{o}{(}\PY{n}{Finished} \PY{n}{st0\PYZsq{}} \PY{n}{ret}\PY{o}{,} \PY{n}{tr0}\PY{o}{)} \PY{o}{\PYZhy{}\PYZgt{}}
    \PY{n}{equivalent\PYZus{}for\PYZus{}principal} \PY{n}{u} \PY{n}{st0} \PY{n}{st1} \PY{o}{\PYZhy{}\PYZgt{}}
    \PY{k}{forall} \PY{n}{st1\PYZsq{}} \PY{n}{tr1}\PY{o}{,}
      \PY{n}{exec} \PY{n}{p} \PY{n}{u} \PY{n}{st1} \PY{n}{rand} \PY{o}{=}
          \PY{n}{Some} \PY{o}{(}\PY{n}{Finished} \PY{n}{st1\PYZsq{}} \PY{n}{ret\PYZsq{}}\PY{o}{,} \PY{n}{tr1}\PY{o}{)} \PY{o}{\PYZhy{}\PYZgt{}}
      \PY{n}{ret\PYZsq{}} \PY{o}{=} \PY{n}{ret}\PY{o}{.}
\end{BVerbatim}

  \caption{Definition of return-value noninterference, capturing that
    return values do not leak other users' confidential data.}
  \label{fig:two-safety-ret}
\end{figure}

The definition of return-value noninterference captures the intuition
about the adversary not being able to learn information about confidential
data: the return value obtained by the adversary by running some
code does not depend on the confidential data.  To make this precise,
\cc{ret\_noninterference} of procedure \cc{p} considers pairs of states,
\cc{st0} and \cc{st1}, which are equivalent as far as some principal
\cc{u} is concerned.  Here, \cc{u} is representing the adversary, and
confidential data is represented by the difference between \cc{st0} and
\cc{st1} that the adversary should not be able to observe.  If \cc{u}
runs procedure \cc{p} in state \cc{st0} and gets return value \cc{ret},
then it must also have been possible for the adversary to get the same
return value, \cc{ret}, if he ran \cc{p} in state \cc{st1} instead.


\paragraph{Defining state noninterference.}

\autoref{fig:two-safety-state} presents \sys's definition
of state noninterference, which complements return-value
noninterference.  This definition helps \sys deal with the indirect-disclosure
challenge from \autoref{s:goal:chal}.  This definition
considers two principals: a \cc{viewer} and a \cc{caller}.  The definition
intuitively says that, by running procedure \cc{p}, the caller will not
create any state differences observable to \cc{viewer}.

\begin{figure}[ht]
  \begin{BVerbatim}[commandchars=\\\{\},codes={\catcode`\$=3\catcode`\^=7\catcode`\_=8},fontsize=\footnotesize]
\PY{k+kn}{Definition} \PY{n}{equiv\PYZus{}state\PYZus{}for\PYZus{}principal} \PY{n}{u} \PY{n}{res0} \PY{n}{res1} \PY{o}{:=}
  \PY{k}{exists} \PY{n}{st0} \PY{n}{st1}\PY{o}{,}
    \PY{n}{equivalent\PYZus{}for\PYZus{}principal} \PY{n}{u} \PY{n}{st0} \PY{n}{st1} \PY{o}{/\PYZbs{}}
    \PY{o}{(}\PY{n}{res0} \PY{o}{=} \PY{n}{Crashed} \PY{n}{st0} \PY{o}{/\PYZbs{}} \PY{n}{res1} \PY{o}{=} \PY{n}{Crashed} \PY{n}{st1} \PY{o}{\PYZbs{}/}
     \PY{k}{exists} \PY{n}{v0} \PY{n}{v1}\PY{o}{,}
       \PY{n}{res0} \PY{o}{=} \PY{n}{Finished} \PY{n}{st0} \PY{n}{v0} \PY{o}{/\PYZbs{}}
\\[0.7\baselineskip]       \PY{n}{res1} \PY{o}{=} \PY{n}{Finished} \PY{n}{st1} \PY{n}{v1}\PY{o}{)}\PY{o}{.}
\PY{k+kn}{Definition} \PY{n}{state\PYZus{}noninterference} \PY{o}{`}\PY{o}{(}\PY{n}{p} \PY{o}{:} \PY{n}{proc} \PY{n}{T}\PY{o}{)} \PY{o}{:=}
  \PY{k}{forall} \PY{n}{viewer} \PY{n}{caller} \PY{n}{st0} \PY{n}{rand} \PY{n}{res0} \PY{n}{tr0} \PY{n}{st1}\PY{o}{,}
    \PY{n}{exec} \PY{n}{p} \PY{n}{caller} \PY{n}{st0} \PY{n}{rand} \PY{o}{=} \PY{n}{Some} \PY{o}{(}\PY{n}{res0}\PY{o}{,} \PY{n}{tr0}\PY{o}{)} \PY{o}{\PYZhy{}\PYZgt{}}
    \PY{n}{equivalent\PYZus{}for\PYZus{}principal} \PY{n}{viewer} \PY{n}{st0} \PY{n}{st1} \PY{o}{\PYZhy{}\PYZgt{}}
    \PY{k}{forall} \PY{n}{res1} \PY{n}{tr1}\PY{o}{,}
      \PY{n}{exec} \PY{n}{p} \PY{n}{caller} \PY{n}{st1} \PY{n}{rand} \PY{o}{=} \PY{n}{Some} \PY{o}{(}\PY{n}{res1}\PY{o}{,} \PY{n}{tr1}\PY{o}{)} \PY{o}{\PYZhy{}\PYZgt{}}
      \PY{n}{equiv\PYZus{}state\PYZus{}for\PYZus{}principal} \PY{n}{viewer} \PY{n}{res0} \PY{n}{res1}\PY{o}{.}
\end{BVerbatim}

  \caption{Definition of state noninterference, capturing that \cc{caller}
    does not indirectly disclose state to \cc{viewer}.}
  \label{fig:two-safety-state}
\end{figure}

More formally, \cc{state\_noninterference} considers two executions
by \cc{caller}, running the same procedure \cc{p}, with the same exact
arguments (encoded inside of \cc{p}).  If the caller runs \cc{p} in two
states that appear equivalent to \cc{viewer}, then the resulting states
in \cc{res0} and \cc{res1} will still appear equivalent to \cc{viewer}.
This definition includes the possibility of a crash while running \cc{p}.

\section{Proof approach: sealed blocks}
\label{s:proof}

Proving that every system call in a file system satisfies
\cc{ret\_noninterference} and \cc{state\_noninterference} would require
a proof that reasons about two executions, which is complex.  To reduce
proof effort, \sys introduces an implementation and proof approach
called \emph{sealed blocks}.  This approach factors out reasoning
about confidentiality of files from most of the file-system logic,
by reasoning about the confidentiality of disk blocks.  The intuition
behind this approach is threefold.  First, all confidential data lives
in file blocks.  Second, the file system itself rarely needs to look
inside of the file blocks.  Finally, permissions on files translate
directly into permissions on the underlying blocks comprising the file.

\begin{figure}[ht]
  \centering
  \scalebox{0.8}{
    \begin{tikzpicture}[>=latex]

  \tikzstyle{block}=[rectangle, draw, minimum height=0.5cm, minimum width=1.5cm, node distance=1cm];
  \tikzstyle{pblock}=[rectangle split, rectangle split parts=2, draw, minimum height=1cm, minimum width=1.5cm, node distance=1cm, align=left];
  \tikzstyle{sealed}=[fill=lightgray]

  \draw node (b0) [block] {data};
  \draw node (b1) [block, right of=b0, xshift=.5cm] {data};
  \draw node (b2) [block, right of=b1, xshift=.5cm] {data};
  \draw node [left of=b0, xshift=-2.5cm, anchor=west] {Real disk:};

  \draw node (p0) [pblock, above of=b0] {block \nodepart{second} perm};
  \draw node (p1) [pblock, right of=p0, xshift=.5cm] {block \nodepart{second} perm};
  \draw node (p2) [pblock, right of=p1, xshift=.5cm] {block \nodepart{second} perm};
  \draw node [left of=p0, xshift=-2.5cm, anchor=west] {Logical disk:};

  \draw [->] (b0) -- (p0);
  \draw [<-] (b2) -- (p2);

  \draw node (fs)
    [ rectangle, rounded corners=.2cm, draw,
      minimum width=6cm, minimum height=2cm,
      above of=p1, yshift=1.5cm ] {};
  \draw node [left of=fs, xshift=-4cm, anchor=west] {File system:};

  \draw node (fsp0) [pblock, sealed, above of=p0, yshift=1.5cm] {block \nodepart{second} perm};
  \draw node (fsp2) [pblock, sealed, above of=p2, yshift=1.5cm] {block \nodepart{second} perm};

  \draw [->] (p0.north) -- (fsp0.south) node [midway, left, align=right]
    {\phantom{x} \\ \em Read \\ \em block};
  \draw [->] (fsp2.south) -- (p2.north) node [midway, right, align=left]
    {\phantom{x} \\ \em Write \\ \em block};

  \draw node (w0) [block, above of=fsp0, yshift=2.25cm] {block};
  \draw node (w2) [block, above of=fsp2, yshift=2.25cm] {block};

  \draw [<-] (w0) -- (fsp0);
  \draw [<-] (fsp2) -- (w2);

  \draw node (fsread)
    [ rectangle, rounded corners=.2cm, draw, fill=white,
      minimum width=2.5cm, minimum height=1.5cm,
      above of=fs, yshift=1cm, xshift=-1.75cm ]
    {
      \begin{minipage}{2.5cm}
        \begin{Verbatim}[commandchars=\\\{\},codes={\catcode`\$=3\catcode`\^=7\catcode`\_=8},fontsize=\scriptsize]
\PY{k}{def} \PY{n+nf}{read}\PY{p}{(}\PY{o}{.}\PY{o}{.}\PY{o}{.}\PY{p}{)}\PY{p}{:}
  \PY{k}{if} \PY{n}{can\PYZus{}access}\PY{p}{(}\PY{p}{)}\PY{p}{:}
    \PY{n}{unseal}\PY{p}{(}\PY{n}{block}\PY{p}{)}
  \PY{o}{.}\PY{o}{.}\PY{o}{.}
\end{Verbatim}

      \end{minipage}
    };

  \draw node (fswrite)
    [ rectangle, rounded corners=.2cm, draw, fill=white,
      minimum width=2.5cm, minimum height=1.5cm,
      above of=fs, yshift=1cm, xshift=1.75cm ]
    {
      \begin{minipage}{2.5cm}
        \begin{Verbatim}[commandchars=\\\{\},codes={\catcode`\$=3\catcode`\^=7\catcode`\_=8},fontsize=\scriptsize]
\PY{k}{def} \PY{n+nf}{write}\PY{p}{(}\PY{o}{.}\PY{o}{.}\PY{o}{.}\PY{p}{)}\PY{p}{:}
  \PY{n}{perm} \PY{o}{=} \PY{n+nb}{file}\PY{o}{.}\PY{n}{acl}
  \PY{n}{seal}\PY{p}{(}\PY{n}{block}\PY{p}{,} \PY{n}{perm}\PY{p}{)}
  \PY{o}{.}\PY{o}{.}\PY{o}{.}
\end{Verbatim}

      \end{minipage}
    };

  \draw node [left of=fsread, xshift=-2.25cm, anchor=west, align=left] {Syscall \\ wrappers:};

\end{tikzpicture}

  }
  \caption{Overview of \sys's proof approach using sealed blocks.}
  \label{fig:sealing}
\end{figure}

\autoref{fig:sealing} presents an overview of \sys's block-sealing
approach.  There are three parts to the block-sealing approach.  The first
is to create a logical disk where every disk block is associated with a
\emph{permission}, which defines the set of principals that can access
this block.  Some permissions are public, indicating that the block is
accessible to anyone.  Other permissions might restrict access to some
users, indicating that this block is storing confidential file data.
\sys is agnostic to the specific choice of principals or permissions;
that is, all of \sys is parameterized over arbitrary types for
principals and permissions.  The logical disk is purely a proof strategy
and does not appear at runtime; the real disk, shown at the bottom of
\autoref{fig:sealing}, has no permissions.

The second part is a sealed-block abstraction, indicated by shaded
blocks in \autoref{fig:sealing}.  A sealed block represents the raw
block contents and the associated permission, but the file system cannot
directly access a sealed block's contents.  Instead, the file-system
implementation must explicitly call \cc{seal()} and \cc{unseal()} to
translate between sealed blocks and their raw contents.  These \cc{seal()}
and \cc{unseal()} functions are also purely part of the proof and do
not appear at runtime.

The code of the file system can read and write arbitrary blocks on
disk, but the result of a read is a sealed block that must be explicitly unsealed
if needed.  The file-system internals can unseal public blocks (e.g.,
containing allocator bitmaps or inodes) but cannot unseal private blocks.
This avoids the need to reason about the file-system implementation when
proving confidentiality, because the file-system implementation never
has access to confidential data.

The third part is the wrappers for system calls that handle confidential
data, namely, \cc{read()} and \cc{write()}.  These wrappers are responsible
for explicitly calling \cc{seal()} and \cc{unseal()} to translate between
the raw data seen by the user (on top of the system call) and the sealed
blocks that are handled in the rest of the file-system implementation.

\sys's sealed-block approach is a good fit for the challenges outlined in
\autoref{s:goal:chal}.  Specifically, there are very few places where a file
system must access the actual contents of a file's disk block---namely, in the
wrappers for the \cc{read()} and \cc{write()} syscalls.  As a result, most
specifications in a file system remain largely the same.  The key difference is
that the specifications promise that the procedure in question does not look
inside of any confidential blocks.  This means that any nondeterminism present
in the specification cannot be used to leak confidential data.

This approach allows file-system developers to avoid proving explicit
confidentiality theorems for most of the file system, but it still allows
\sys to conclude that confidentiality is not violated.  \sys provides a
theorem that proves two-safety for any file-system implementation that
correctly uses the sealed-block abstraction.  As a result, the
file-system developer need not reason about complex two-safety theorems
and can limit their reasoning to single executions.


\subsection{Formalizing sealed blocks}

To formally define \sys's sealed-block abstraction, \sys uses the
notion of a \emph{handle} to represent a sealed block.  \sys requires
the developer to perform two steps.  The first is to modify their code
to use the sealed-block abstraction: that is, to pass around handles
for blocks and to call \cc{seal()} and \cc{unseal()} as necessary.
The second is to prove that their code correctly follows the unsealing
rules.  This boils down to ensuring that sealed blocks are unsealed only
when the principal has appropriate permission for that block.

\sys models this by extending traditional Hoare logic to reason about
unseal operations.  Specifically, \sys builds on CHL~\cite{chen:fscq},
where functional correctness specifications are written in terms of
pre- and postconditions.  \sys, first, extends the execution semantics
(as we describe next) to produce an \emph{unseal trace} consisting of
unseal operations and, second, extends the specifications to require
that the unseal trace contain only allowed unseals.

We expect that systems built on top of \sys would often group multiple
blocks into a single object (e.g., multiple blocks comprising a single
file in a file system).  To help developers reason about all of these
blocks sharing the same permissions, \sys introduces the notion of
a \emph{domain}.  This is a layer of indirection between blocks and
permissions.  Specifically, sealed blocks point to a domain ID (e.g.,
an inode number in the case of a file system), and the domain in turn
specifies the permission for those blocks (e.g., the permission reflected
in the inode's data structure).



\paragraph{Execution model.}

\sys's execution model requires the implementation to be written in a
domain-specific language, based on CHL and implemented inside of Coq,
which provides several primitive operations.  These operations include
reading and writing the disk, manipulating sealed blocks by sealing and
unsealing, as well as others for sequencing computation, returning values,
flushing disk writes, etc.

\begin{figure}[ht]
  \begin{BVerbatim}[commandchars=\\\{\},codes={\catcode`\$=3\catcode`\^=7\catcode`\_=8},fontsize=\footnotesize]
\PY{k+kn}{Inductive} \PY{n}{nondet\PYZus{}decision} \PY{o}{:=}
\PY{o}{|} \PY{n}{RandomHandle} \PY{o}{(}\PY{n}{h}\PY{o}{:}\PY{n}{handle}\PY{o}{)}
\PY{c}{(*}\PY{c}{ Other types of non\PYZhy{}determinism omitted for space }\PY{c}{*)}
\\[0.7\baselineskip]\PY{o}{|} \PY{n}{CrashHere}\PY{o}{.}
\\[0.7\baselineskip]\PY{k+kn}{Definition} \PY{n}{oracle} \PY{o}{:=} \PY{k+kt}{list} \PY{n}{nondet\PYZus{}decision}\PY{o}{.}
\PY{k+kn}{Definition} \PY{n}{exec} \PY{o}{`}\PY{o}{(}\PY{n}{code}\PY{o}{:}\PY{n}{proc} \PY{n}{T}\PY{o}{)} \PY{o}{(}\PY{n}{u}\PY{o}{:}\PY{n}{Principal}\PY{o}{)} \PY{o}{(}\PY{n}{st}\PY{o}{:}\PY{n}{State}\PY{o}{)}
           \PY{o}{(}\PY{n}{rand}\PY{o}{:}\PY{n}{oracle}\PY{o}{)} \PY{o}{:} \PY{n}{option} \PY{o}{(}\PY{n}{result} \PY{n}{T} \PY{o}{*} \PY{n}{trace}\PY{o}{)} \PY{o}{:=}
  \PY{k}{match} \PY{n}{code}\PY{o}{,} \PY{n}{rand} \PY{k}{with}
  \PY{o}{|} \PY{n}{ChangePerm} \PY{o}{\PYZus{}} \PY{o}{\PYZus{}}\PY{o}{,} \PY{n}{CrashHere} \PY{o}{=\PYZgt{}} \PY{n}{None}
  \PY{o}{|} \PY{o}{\PYZus{}}\PY{o}{,} \PY{n}{CrashHere} \PY{o}{=\PYZgt{}} \PY{n}{Some} \PY{o}{(}\PY{n}{Crashed} \PY{n}{st}\PY{o}{,} \PY{n+nb+bp}{[]}\PY{o}{)}
  \PY{o}{|} \PY{n}{Read} \PY{n}{a}\PY{o}{,} \PY{n}{RandomHandle} \PY{n}{h} \PY{o}{=\PYZgt{}}
    \PY{k}{if} \PY{n}{addr\PYZus{}out\PYZus{}of\PYZus{}bounds} \PY{n}{st} \PY{n}{a} \PY{k}{then}
      \PY{n}{Some} \PY{o}{(}\PY{n}{Finished} \PY{n}{st} \PY{n}{hzero}\PY{o}{,} \PY{n+nb+bp}{[]}\PY{o}{)}
    \PY{k}{else} \PY{k}{if} \PY{n}{handle\PYZus{}used} \PY{n}{st} \PY{n}{h} \PY{k}{then}
      \PY{n}{None}
    \PY{k}{else}
      \PY{k}{let} \PY{n}{data} \PY{o}{:=} \PY{n}{disk\PYZus{}block\PYZus{}data} \PY{n}{st} \PY{n}{a} \PY{k}{in}
      \PY{k}{let} \PY{n}{dom}  \PY{o}{:=} \PY{n}{disk\PYZus{}block\PYZus{}dom}  \PY{n}{st} \PY{n}{a} \PY{k}{in}
      \PY{k}{let} \PY{n}{st\PYZsq{}} \PY{o}{:=} \PY{n}{install\PYZus{}handle} \PY{n}{st} \PY{n}{h} \PY{o}{(}\PY{n}{data}\PY{o}{,} \PY{n}{dom}\PY{o}{)} \PY{k}{in}
      \PY{n}{Some} \PY{o}{(}\PY{n}{Finished} \PY{n}{st\PYZsq{}} \PY{n}{h}\PY{o}{,} \PY{n+nb+bp}{[]}\PY{o}{)}
  \PY{o}{|} \PY{n}{Write} \PY{n}{a} \PY{n}{h}\PY{o}{,} \PY{o}{\PYZus{}} \PY{o}{=\PYZgt{}}
    \PY{k}{if} \PY{n}{handle\PYZus{}used} \PY{n}{st} \PY{n}{h} \PY{k}{then}
      \PY{k}{let} \PY{n}{data} \PY{o}{:=} \PY{n}{handle\PYZus{}data} \PY{n}{st} \PY{n}{h} \PY{k}{in}
      \PY{k}{let} \PY{n}{dom}  \PY{o}{:=} \PY{n}{handle\PYZus{}dom}  \PY{n}{st} \PY{n}{h} \PY{k}{in}
      \PY{k}{let} \PY{n}{st\PYZsq{}} \PY{o}{:=} \PY{n}{disk\PYZus{}block\PYZus{}write} \PY{n}{st} \PY{n}{a} \PY{o}{(}\PY{n}{data}\PY{o}{,} \PY{n}{dom}\PY{o}{)} \PY{k}{in}
      \PY{n}{Some} \PY{o}{(}\PY{n}{Finished} \PY{n}{st\PYZsq{}} \PY{n}{tt}\PY{o}{,} \PY{n+nb+bp}{[]}\PY{o}{)}
    \PY{k}{else}
      \PY{n}{Some} \PY{o}{(}\PY{n}{Finished} \PY{n}{st} \PY{n}{tt}\PY{o}{,} \PY{n+nb+bp}{[]}\PY{o}{)}
  \PY{c}{(*}\PY{c}{ Some transitions omitted for space reasons }\PY{c}{*)}
  \PY{o}{|} \PY{n}{Seal} \PY{n}{data} \PY{n}{dom}\PY{o}{,} \PY{n}{RandomHandle} \PY{n}{h} \PY{o}{=\PYZgt{}}
    \PY{k}{if} \PY{n}{handle\PYZus{}used} \PY{n}{st} \PY{n}{h} \PY{k}{then}
      \PY{n}{None}
    \PY{k}{else}
      \PY{k}{let} \PY{n}{st\PYZsq{}} \PY{o}{:=} \PY{n}{install\PYZus{}handle} \PY{n}{st} \PY{n}{h} \PY{o}{(}\PY{n}{data}\PY{o}{,} \PY{n}{dom}\PY{o}{)} \PY{k}{in}
      \PY{n}{Some} \PY{o}{(}\PY{n}{Finished} \PY{n}{st\PYZsq{}} \PY{n}{h}\PY{o}{,} \PY{n+nb+bp}{[]}\PY{o}{)}
  \PY{o}{|} \PY{n}{Unseal} \PY{n}{h}\PY{o}{,} \PY{o}{\PYZus{}} \PY{o}{=\PYZgt{}}
    \PY{k}{if} \PY{n}{handle\PYZus{}used} \PY{n}{st} \PY{n}{h} \PY{k}{then}
      \PY{k}{let} \PY{n}{data} \PY{o}{:=} \PY{n}{handle\PYZus{}data} \PY{n}{st} \PY{n}{h} \PY{k}{in}
      \PY{k}{let} \PY{n}{dom}  \PY{o}{:=} \PY{n}{handle\PYZus{}dom}  \PY{n}{st} \PY{n}{h} \PY{k}{in}
      \PY{k}{let} \PY{n}{perm} \PY{o}{:=} \PY{n}{domain\PYZus{}perm} \PY{n}{st} \PY{n}{dom} \PY{k}{in}
      \PY{n}{Some} \PY{o}{(}\PY{n}{Finished} \PY{n}{st} \PY{n}{data}\PY{o}{,} \PY{o}{[}\PY{n}{perm}\PY{o}{]}\PY{o}{)}
    \PY{k}{else}
      \PY{n}{Some} \PY{o}{(}\PY{n}{Finished} \PY{n}{st} \PY{n}{zero}\PY{o}{,} \PY{n+nb+bp}{[]}\PY{o}{)}
  \PY{o}{|} \PY{n}{ChangePerm} \PY{n}{dom} \PY{n}{newperm}\PY{o}{,} \PY{o}{\PYZus{}} \PY{o}{=\PYZgt{}}
    \PY{k}{let} \PY{n}{oldperm} \PY{o}{:=} \PY{n}{domain\PYZus{}perm} \PY{n}{st} \PY{n}{dom} \PY{k}{in}
    \PY{k}{let} \PY{n}{st\PYZsq{}} \PY{o}{:=} \PY{n}{domain\PYZus{}set\PYZus{}perm} \PY{n}{st} \PY{n}{dom} \PY{n}{newperm} \PY{k}{in}
    \PY{n}{Some} \PY{o}{(}\PY{n}{Finished} \PY{n}{st} \PY{n}{tt}\PY{o}{,} \PY{o}{[}\PY{n}{oldperm}\PY{o}{]}\PY{o}{)}
  \PY{o}{|} \PY{o}{\PYZus{}}\PY{o}{,} \PY{o}{\PYZus{}} \PY{o}{=\PYZgt{}} \PY{n}{None}
  \PY{k}{end}\PY{o}{.}
\end{BVerbatim}

  \caption{Execution semantics with logging of unseal operations.}
  \label{fig:exec}
\end{figure}

\autoref{fig:exec} shows a simplified version of \sys's execution
semantics.  The semantics are defined as a function that takes the
\cc{code} being executed (of type \cc{proc T}), the principal \cc{u}
running the operation (of type \cc{Principal}), and the starting state \cc{st}
(of type \cc{State}).  The function
produces a tuple consisting of a result (of type \cc{result T}) and
a trace of unsealed permissions (of type \cc{trace}).  The function
is allowed to return \cc{None} (as indicated by the \cc{option} type)
when there is no execution possible for the supplied randomness (e.g.,
the randomly chosen handle is already in use).

For example, consider the case that handles the \cc{Read a} operation,
which describes the execution of reading address \cc{a} from disk.
There are three sub-cases.  If the address is out of bounds, the \cc{Read}
returns a handle for a zero block, with an empty unseal trace.  If the handle
\cc{h} supplied by the randomness oracle is already in use, no execution
is possible.  Otherwise, the \cc{Read} initializes the new handle to
represent the block from address \cc{a}, with the block's domain ID, and
returns that handle, with an empty trace because no blocks were unsealed.

As another example, the \cc{Unseal h} operation produces a nonempty trace,
consisting of the permission of the sealed block whose handle \cc{h}
was unsealed, as long as the handle was valid (otherwise, \cc{Unseal}
returns zero).  Since the sealed block points to a domain ID, \cc{dom},
the semantics of \cc{Unseal} look up the corresponding permissions of
that domain.  One omitted rule handles concatenation of unseal traces when a
developer sequences one statement after another.

The \cc{ChangePerm dom newperm} operation allows the developer to
change permissions of a domain.  This operation is used in implementing
\cc{chown}.  The semantics of \cc{ChangePerm} modify the permission
associated with the domain, and produce an unseal trace containing the domain's
old permission, to reflect that data with that permission may have
been disclosed.  Since the domains are purely a proof construct,
\cc{ChangePerm} is a purely logical operation, which does not perform
any actions at runtime.

%%% Rewrite this to remove oracles
{\color{red} Rewrite this to remove oracles. }\\
Finally, \cc{exec} describes the possible crash behaviors of the system.
For example, the case for \cc{\_, CrashHere} states that it is possible to
crash in the starting state, regardless of what code was being executed,
if the randomness oracle tells us \cc{CrashHere}.  A combination
of other rules, not shown, allow crashing in the middle of a sequence of
operations.  The very first case, for \cc{ChangePerm \_ \_, CrashHere},
says that \cc{ChangePerm} cannot crash.  This reflects the fact that
\cc{ChangePerm} is a purely logical operation.

{\color{red} Rewrite Ends Here. }\\
%%%%%%%%%%%%%%%%%%%%%%%%%%%%%%%

\paragraph{Specification and verification of unseal rules.}

\sys requires developers to write specifications for each procedure, using
pre- and postconditions.  The postcondition describes how the procedure
modifies the state of the system, along with what must be true of the procedure's
return value, assuming that the precondition (a predicate over the system
state and the procedure's arguments) held at the start of the procedure.

To reason about what blocks a procedure might unseal, \sys augments
specification postconditions with requirements about the permissions
that appear in the unseal trace produced by the execution of the procedure.

\begin{figure}[ht]
  \begin{BVerbatim}[commandchars=\\\{\},codes={\catcode`\$=3\catcode`\^=7\catcode`\_=8},fontsize=\footnotesize]
\PY{k+kn}{Definition} \PY{n}{unseal\PYZus{}safe} \PY{o}{`}\PY{o}{(}\PY{n}{p} \PY{o}{:} \PY{n}{proc} \PY{n}{T}\PY{o}{)} \PY{o}{:=}
  \PY{k}{forall} \PY{n}{u} \PY{n}{st} \PY{n}{rand} \PY{n}{res} \PY{n}{tr}\PY{o}{,}
    \PY{n}{exec} \PY{n}{p} \PY{n}{u} \PY{n}{st} \PY{n}{rand} \PY{o}{=} \PY{n}{Some} \PY{o}{(}\PY{n}{res}\PY{o}{,} \PY{n}{tr}\PY{o}{)} \PY{o}{\PYZhy{}\PYZgt{}}
      \PY{k}{forall} \PY{n}{perm}\PY{o}{,}
        \PY{n}{In} \PY{n}{perm} \PY{n}{tr} \PY{o}{\PYZhy{}\PYZgt{}} \PY{n}{can\PYZus{}access} \PY{n}{u} \PY{n}{perm}\PY{o}{.}
\end{BVerbatim}

  \caption{Definition of unseal safety.}
  \label{fig:unseal-safety}
\end{figure}

\autoref{fig:unseal-safety} shows \sys's definition of unseal safety.
This definition says that procedure \cc{p} is ``unseal-safe'' if,
for every principal \cc{u} that runs this procedure and any starting
state \cc{st}, all permissions produced by this procedure in its unseal trace
\cc{tr} will be accessible to the calling principal.
%(This definition
%is simplified from \sys's actual implementation, because \sys's real
%definition appears in the postcondition of a more complex
%functional-correctness statement about all possible executions of \cc{p}.)
Proving unseal safety leads to a proof obligation for the file-system
developer---namely, proving that the implementation will unseal a block
only if the current principal has access to it.

File-system implementation code falls into three categories with respect
to proving unseal safety.  The first category are procedures that do not
invoke any \cc{Unseal} operations.  For these procedures, the resulting
unseal trace is always empty, and \sys is able to prove unseal safety without
any developer input.  Most of the file-system code falls in this category.

The second category are procedures that unseal public blocks.
Examples include accessing inodes, allocator bitmaps, directories, etc.
These procedures do produce unseal traces containing permissions, but all
of the permissions should be public.  Thus, the developer's job is to
show that these permissions are indeed public; once this is established,
showing that the current principal has access is straightforward (since
every principal has access to public permissions).

To prove that the permissions are indeed public, the developer
relies on representation invariants of the file system.  For example, the
invariant for the block allocator states that all of the bitmap blocks
are public. The developer can assume this invariant within any implementation of
the block allocator API, which helps her prove that the block in question has
public permissions. In turn the developer must prove that the invariant is
preserved by every procedure (including across crashes and recovery), and show
that it is established at initialization time by \cc{mkfs}.

The final category are procedures that unseal private blocks.  In a
file system, this happens only in the implementation of the \cc{read}
system call, which returns file data to the caller.  The implementation (wrapper)
of the \cc{read} system call contains explicit code to obtain the current
principal, get the file's ACL (access control list) from the inode, and
compare them.  The developer's job is to prove that this code correctly
performs the permission check.  This proof typically relies on the
file's representation invariant, which asserts that every file block
is tagged with a permission matching the ACL stored in the inode.

\begin{figure}[ht]
  \begin{BVerbatim}[commandchars=\\\{\},codes={\catcode`\$=3\catcode`\^=7\catcode`\_=8},fontsize=\footnotesize]
\PY{k+kn}{Definition} \PY{n}{unseal\PYZus{}public} \PY{o}{`}\PY{o}{(}\PY{n}{p} \PY{o}{:} \PY{n}{proc} \PY{n}{T}\PY{o}{)} \PY{o}{:=}
  \PY{k}{forall} \PY{n}{u} \PY{n}{st} \PY{n}{rand} \PY{n}{res} \PY{n}{tr}\PY{o}{,}
    \PY{n}{exec} \PY{n}{p} \PY{n}{u} \PY{n}{st} \PY{n}{rand} \PY{o}{=} \PY{n}{Some} \PY{o}{(}\PY{n}{res}\PY{o}{,} \PY{n}{tr}\PY{o}{)} \PY{o}{\PYZhy{}\PYZgt{}}
      \PY{k}{forall} \PY{n}{perm}\PY{o}{,}
        \PY{n}{In} \PY{n}{perm} \PY{n}{tr} \PY{o}{\PYZhy{}\PYZgt{}} \PY{n}{perm} \PY{o}{=} \PY{n}{Public}\PY{o}{.}
\end{BVerbatim}

  \caption{Definition of \cc{unseal\_public}.}
  \label{fig:unseal-public}
\end{figure}

\sys also provides a stronger version of unseal-safety, as shown in
\autoref{fig:unseal-public}, called \cc{unseal\_public}.  A procedure
satisfies this definition if all of its code falls in the first two
categories above: that is, the procedure either unseals no blocks
at all or unseals only public blocks.  This alternative definition is
strictly stronger than unseal-safety; any procedure that satisfies
\cc{unseal\_public} is also unseal-safe.  The distinction between these
two notions will help the developer prove noninterference theorems,
as we will describe in \autoref{s:design:prove}.

\paragraph{Crashes.}

\sys's approach naturally extends to reasoning about crashes.
\sys's disk-crash model builds on the CHL model of disk
crashes~\cite{chen:fscq, chen:dfscq}.  After a crash, disk blocks can
be updated nondeterministically, as in CHL, based on outstanding writes
that are in the disk's write buffer but have not been flushed yet to
durable storage.  However, domains always follow the
data for pending writes; that is, logically, the content of the disk
block is updated atomically together with its domain ID\@.

All handles are invalidated after a crash, to model the fact that the
computer reboots and all in-memory state is lost.  All recovery code,
such as log replay or \cc{fsck}, is proven correct in \sys, which means
that it must follow the same block-sealing rules as the rest of the
file-system code.  This ensures that no data can be disclosed by the
recovery code.


\subsection{Proving noninterference}
\label{s:design:prove}

To help the developer prove the two types of noninterference, \sys
provides helper theorems.  \autoref{fig:unseal-to-ret} shows the first
one, which proves return-value noninterference based on unseal-safety.
\sys proves this theorem by considering all operations performed by
procedure \cc{p}.  Each operation must produce the same result in the
two executions being considered, since the states are equivalent for the
principal in question, \cc{u}.  The only way in which the executions
could differ is if they unsealed a block that was not accessible
to \cc{u}.  However, \cc{unseal\_safe} says that this is impossible.
This theorem also applies to procedures that are \cc{unseal\_public},
since that notion is strictly stronger than \cc{unseal\_safe}.

\begin{figure}[ht]
  \begin{BVerbatim}[commandchars=\\\{\},codes={\catcode`\$=3\catcode`\^=7\catcode`\_=8},fontsize=\footnotesize]
\PY{k+kn}{Theorem} \PY{n}{unseal\PYZus{}safe\PYZus{}to\PYZus{}ret\PYZus{}noninterference} \PY{o}{:}
  \PY{k}{forall} \PY{o}{`}\PY{o}{(}\PY{n}{p} \PY{o}{:} \PY{n}{proc} \PY{n}{T}\PY{o}{)}\PY{o}{,}
    \PY{n}{unseal\PYZus{}safe} \PY{n}{p} \PY{o}{\PYZhy{}\PYZgt{}} \PY{n}{ret\PYZus{}noninterference} \PY{n}{p}\PY{o}{.}
\end{BVerbatim}

  \caption{Theorem connecting unseal-safety to return-value noninterference.}
  \label{fig:unseal-to-ret}
\end{figure}

\autoref{fig:unseal-to-state} shows the second theorem provided by \sys,
for reasoning about state noninterference.  This theorem requires that
the procedure satisfy the stronger definition, \cc{unseal\_public}, to
ensure state noninterference.  The intuition for why this theorem is
true lies in the fact that a procedure that unseals only public blocks
cannot obtain any confidential data in the first place.  As a result,
this procedure's execution will be identical regardless of the contents of
confidential blocks, and thus the state after this procedure's execution
will remain equivalent from the adversary's point of view.  \sys proves
this theorem formally in Coq.

\begin{figure}[ht]
  \begin{BVerbatim}[commandchars=\\\{\},codes={\catcode`\$=3\catcode`\^=7\catcode`\_=8},fontsize=\footnotesize]
\PY{k+kn}{Theorem} \PY{n}{unseal\PYZus{}public\PYZus{}to\PYZus{}state\PYZus{}noninterference} \PY{o}{:}
  \PY{k}{forall} \PY{o}{`}\PY{o}{(}\PY{n}{p} \PY{o}{:} \PY{n}{proc} \PY{n}{T}\PY{o}{)}\PY{o}{,}
    \PY{n}{unseal\PYZus{}public} \PY{n}{p} \PY{o}{\PYZhy{}\PYZgt{}} \PY{n}{state\PYZus{}noninterference} \PY{n}{p}\PY{o}{.}
\end{BVerbatim}

  \caption{Theorem connecting \cc{unseal\_public} to state noninterference.}
  \label{fig:unseal-to-state}
\end{figure}

\sys does not provide a general-purpose theorem for reasoning about state
noninterference for procedures that satisfy only the weaker notion of
unseal-safety (i.e., that unseal private blocks), such as the \cc{read()}
system call.  Such procedures can indirectly disclose data as
described in \autoref{s:goal:chal} to legitimately unseal confidential
data on behalf of the currently executing principal but then stash a
copy of it.  It is up to the file-system developer to prove the state
noninterference of those procedures.  \autoref{s:fs} will discuss in
more detail how \sfscq structures its implementation to simplify these
proofs; in the case of \sfscq, the only system call that requires this
type of reasoning is \cc{read}.
