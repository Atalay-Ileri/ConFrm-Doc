\chapter{Case study: File system}
\label{s:fs}

To evaluate whether Disksec allows specifying and proving confidentiality
for a file system, we applied Disksec to the DFSCQ verified file system,
producing the SFSCQ verified secure file system, as described below.


\subsection{Specifying security}

The core specification of confidentiality for SFSCQ lies in the
write system call, as shown in XXX.  This
specification says that the \emph{data} argument to the {write}
system call remains confidential.  This is stated formally by
considering two different executions, starting from the same
state {st}, where different data ({data0} and {data1})
are written to the same offset {off} of the same file {f}.
The results, {res0} and {res1}, must be equivalent for any
adversary {adv} that does not have permission to access file {f}.
Since {equivalent\_state\_for\_principal} considers both crashing and
noncrashing executions, this definition ensures that the data passed to
{write} remains confidential regardless of whether the system crashes
or not.

\begin{figure}[ht]
  \begin{BVerbatim}[commandchars=\\\{\},codes={\catcode`\$=3\catcode`\^=7\catcode`\_=8},fontsize=\footnotesize]
\PY{k+kn}{Theorem} \PY{n}{write\PYZus{}confidentiality} \PY{o}{:}
  \PY{k}{forall} \PY{n}{f} \PY{n}{off} \PY{n}{data0} \PY{n}{data1} \PY{n}{caller} \PY{n}{st} \PY{n}{rand} \PY{n}{res0} \PY{n}{tr0}\PY{o}{,}
    \PY{n}{exec} \PY{o}{(}\PY{n}{write} \PY{n}{f} \PY{n}{off} \PY{n}{data0}\PY{o}{)} \PY{n}{caller} \PY{n}{st} \PY{n}{rand} \PY{o}{=}
                               \PY{n}{Some} \PY{o}{(}\PY{n}{res0}\PY{o}{,} \PY{n}{tr0}\PY{o}{)} \PY{o}{\PYZhy{}\PYZgt{}}
      \PY{k}{exists} \PY{n}{res1} \PY{n}{tr1}\PY{o}{,}
        \PY{n}{exec} \PY{o}{(}\PY{n}{write} \PY{n}{f} \PY{n}{off} \PY{n}{data1}\PY{o}{)} \PY{n}{caller} \PY{n}{st} \PY{n}{rand} \PY{o}{=}
                               \PY{n}{Some} \PY{o}{(}\PY{n}{res1}\PY{o}{,} \PY{n}{tr1}\PY{o}{)} \PY{o}{/\PYZbs{}}
        \PY{k}{forall} \PY{n}{adv}\PY{o}{,}
          \PY{o}{\PYZti{}} \PY{n}{can\PYZus{}access} \PY{n}{adv} \PY{o}{(}\PY{n}{file\PYZus{}perm} \PY{n}{st} \PY{n}{f}\PY{o}{)} \PY{o}{\PYZhy{}\PYZgt{}}
          \PY{n}{equiv\PYZus{}state\PYZus{}for\PYZus{}principal} \PY{n}{adv} \PY{n}{res0} \PY{n}{res1}\PY{o}{.}
\end{BVerbatim}

  \caption{Confidentiality specification for the \cc{write} system call.}
  \label{fig:writespec}
\end{figure}

The other part of the security specification lies in the {chown}
system call, which changes the permissions on existing files, and thus
affects what data is or is not confidential.  Because {chown} can
disclose the contents of a previously confidential file, the standard
definition of state non-interference from XXX
does not hold for {chown}.  Specifically, even if an adversary
{viewer} could not distinguish states {st0} and {st1} before some
{caller} executed {chown}, the adversary may nonetheless be able
to distinguish {st0} and {st1} after the {chown} runs because
the adversary may now have permission to read the previously confidential
file.

The security of {chown} is defined by a specialized version of
state non-interference, which considers three cases.  The first case
is that the adversary {viewer} does not have access to the file
after the {chown} (i.e., is not the new owner).
In this case, state non-interference holds.
The second case is that the adversary {viewer} does gain access to
the file after {chown} (i.e., is the new owner),
but the file had the same contents in the
two executions (i.e., in states {st0} and {st1}).  In this case,
state non-interference holds as well.  Finally, the adversary {viewer}
may gain access to the file \emph{and} the files had different contents in
the two executions.  In this case, state non-interference does not apply.
XXX summarizes this formally.

%\begin{figure}[ht]
%  \begin{BVerbatim}[commandchars=\\\{\},codes={\catcode`\$=3\catcode`\^=7\catcode`\_=8},fontsize=\footnotesize]
\PY{k+kn}{Definition} \PY{n}{chown\PYZus{}state\PYZus{}noninterference} \PY{n}{f} \PY{n}{new\PYZus{}owner} \PY{o}{:=}
  \PY{k}{forall} \PY{n}{viewer} \PY{n}{caller} \PY{n}{st0} \PY{n}{rand} \PY{n}{res0} \PY{n}{tr0} \PY{n}{st1}\PY{o}{,}
    \PY{n}{exec} \PY{o}{(}\PY{n}{chown} \PY{n}{f} \PY{n}{new\PYZus{}owner}\PY{o}{)} \PY{n}{caller} \PY{n}{st0} \PY{n}{rand} \PY{o}{=}
                                \PY{n}{Some} \PY{o}{(}\PY{n}{res0}\PY{o}{,} \PY{n}{tr0}\PY{o}{)} \PY{o}{\PYZhy{}\PYZgt{}}
    \PY{o}{(} \PY{n}{file\PYZus{}data} \PY{n}{st0} \PY{n}{f} \PY{o}{=} \PY{n}{file\PYZus{}data} \PY{n}{st1} \PY{n}{f} \PY{o}{\PYZbs{}/}
      \PY{n}{viewer} \PY{o}{\PYZlt{}}\PY{o}{\PYZgt{}} \PY{n}{new\PYZus{}owner} \PY{o}{)} \PY{o}{\PYZhy{}\PYZgt{}}
    \PY{n}{equivalent\PYZus{}for\PYZus{}principal} \PY{n}{viewer} \PY{n}{st0} \PY{n}{st1} \PY{o}{\PYZhy{}\PYZgt{}}
      \PY{k}{exists} \PY{n}{res1} \PY{n}{tr1}\PY{o}{,}
        \PY{n}{exec} \PY{o}{(}\PY{n}{chown} \PY{n}{f} \PY{n}{new\PYZus{}owner}\PY{o}{)} \PY{n}{caller} \PY{n}{st1} \PY{n}{rand} \PY{o}{=}
                                \PY{n}{Some} \PY{o}{(}\PY{n}{res1}\PY{o}{,} \PY{n}{tr1}\PY{o}{)} \PY{o}{/\PYZbs{}}
        \PY{n}{equiv\PYZus{}state\PYZus{}for\PYZus{}principal} \PY{n}{viewer} \PY{n}{res0} \PY{n}{res1}\PY{o}{.}
\end{BVerbatim}

%  \caption{Confidentiality specification for the {chown} system call.}
%  \label{fig:chownspec}
%\end{figure}

The {write} and {chown} specifications, shown above, are the only
parts of the security specification that are specific to the file system,
because they define where confidential data enters the system in the
first place, and how permissions on that confidential data can change.
Somewhat counter-intuitively, no special treatment is required in the
specifications of other system calls, such as {read}.  Instead, it suffices to
prove the two general noninterference theorems for all system calls
(i.e., {ret\_noninterference} and {state\_noninterference}).
This is because we do not want to consider specific attacks, such as
whether {read} has a missing access-control check.  Instead, Disksec's
noninterference definitions ensure that confidential data cannot be
disclosed regardless of what system calls the adversary tries to use.

Integrity of the file system is a functional-correctness property
and thus is covered by SFSCQ's specifications, alongside other
correctness properties.
Integrity did not require SFSCQ to use any machinery from Disksec for
reasoning about confidential data.


\subsection{Modifying the implementation}
\label{s:fs:impl}


\paragraph{Changing representation invariants.}

DFSCQ consists of many modules, such as the write-ahead log, the bitmap
allocator, the inode module, etc.  Each module has its own invariant
that describes how that module's state is represented in terms of blocks.
For example, the bitmap allocator describes how the free bits are packed
into disk blocks, where they are stored on disk, and the semantics of
each bit.

For SFSCQ, we modified all invariants that describe disk blocks to
state the domain IDs that go along with those blocks.  For instance,
we modified the invariant of the allocator to state that the bitmap
blocks are public.  We modified the write-ahead log layer to expose the
underlying domain IDs on disk blocks to modules implemented on top of
the write-ahead log (in addition to modifying the log invariant to state
that the log metadata is public).

The only nonpublic data is the file contents.  We modified the file
invariant to state that the domain ID of every file block matches the
file's inode number, and the permissions for a particular domain ID match
the ACL stored in the inode with the inode number matching the domain ID\@.

One surprising issue that we encountered came up in the DFSCQ write-ahead
log.  For performance, DFSCQ's write-ahead log used checksums to verify
block contents after a crash.  As a result, the recovery procedure
unsealed blocks from the write-ahead log after a crash, including blocks
that contain confidential data.

To address this issue, we switched to a barrier-based write-ahead log
instead, which is the default design of Linux ext4.  Instead of using
checksums, the barrier-based write-ahead log issues a disk flush between
writing the contents of new log entries and updating the log header.
(DFSCQ already included an implementation of this barrier-based
write-ahead log but did not use it by default.)


\paragraph{Modifying code.}

Loosely speaking, DFSCQ modules handle two kinds of blocks: blocks that
they manipulate (e.g., the bitmap allocator manipulating the bitmap
blocks) and blocks that they pass through (e.g., the write-ahead log
handling reads and writes as part of a transaction, or the file layer
handling file reads and writes).  The first category required a module
to access the block contents, so we added {Seal} and {Unseal}
operations accordingly.  Virtually all operations that fell in this
category involved sealing and unsealing public data.  For the second
category, we did not seal or unseal the data and instead transparently
passed through the handle representing the block; as a result, the module
was oblivious to the domain IDs associated with the disk block.

Private data is sealed and unsealed at the top of the SFSCQ
implementation; that is, in the implementation of the {read}
and {write} system calls.  We modified the {write} system-call
implementation to {Seal} the blocks with the file's inode number
as the domain ID, before processing them further.
We modified the {read} system call to implement the
permission-checking logic---i.e., reading the ACL from the file's inode,
checking whether the currently running principal has access to the file,
and unsealing the block only if the check passes.


\paragraph{Changing intermediate specifications.}

We augmented the Hoare-logic specifications of all internal SFSCQ
procedures to require that the procedure be {unseal\_public}.
This change required little manual effort, because we simply changed
the underlying definition of the Hoare-logic specification to require
{unseal\_public}.  For the write-ahead log, we added additional
constraints in the specification of the {log\_write} procedure,
requiring that the blocks written as part of a transaction must be public,
as described above.


\subsection{Proving security}


\paragraph{Reproving functional correctness.}

Many existing proofs in DFSCQ broke after we made the above changes.  The proofs
broke for three reasons: there were now additional {Seal} and {Unseal}
operations in the code (e.g., the bitmap allocator now sealed and unsealed its
bitmap blocks), the logical representation of a block changed to include a
domain ID, and the specification changed (e.g., augmenting the invariant to
state the domain ID of a block).  This required manually tweaking most of the
proofs to fix them.  The proof changes were simple since the code's logic and
the proof argument remained unchanged.


\paragraph{Proving unsealing.}

In addition to fixing existing proofs, SFSCQ's specifications required us to prove that
the {Unseal} operation was used correctly.  For most procedures, the
specification required that the procedure satisfy {unseal\_public}.
Proving that only public blocks were unsealed required us to demonstrate
that the block was indeed public by referring to the invariant.

For the implementation of the {read} system call, which unseals private
data, we had to prove that {read} correctly implements the permission
check in its code.  This means proving that {read} calls {Unseal}
only after checking permissions, and that the code for the permission
check returns ``allowed'' only if the current principal really does have
permission to access the file contents.  This proof mostly boiled down to
showing that the code implementing the access-control check in {read}
matches the logical permission required by the specification.


\paragraph{Proving noninterference.}

Proving that SFSCQ provides confidentiality required us to
prove three theorems.  The first is that {write} implements the
specification from \autoref{fig:writespec}.  This shows that SFSCQ
will treat data passed by an application to {write} as confidential.
The second is that system calls satisfy {ret\_noninterference}.
This shows that an adversary cannot use any of SFSCQ's system calls
to learn confidential data.  The final is that all system calls satisfy
{state\_noninterference}.  This shows that SFSCQ will not indirectly
leak a user's data when the user invokes an otherwise-benign
system call.  Taken together, these theorems allow an application to
formally conclude that its data remains private, as we show in
\autoref{s:eval}.

Proving {ret\_noninterference} was the easiest, using Disksec's theorem
from \autoref{fig:unseal-to-ret}.  All SFSCQ procedures are proven to be
unseal safe, so no further proof effort is required.

Proving {state\_noninterference} was simple for all system
calls except {read}, because those system calls satisfy {unseal\_public},
allowing us to apply Disksec's theorem from \autoref{fig:unseal-to-state}.
For {read}, we structured the system-call implementation in two parts:
a {read\_helper}, which returns the handle to the data read from
the file, and a wrapper around {read\_helper} that unseals the data
and returns it to the user.  {read\_helper} is {unseal\_public},
allowing us to apply Disksec's theorem from \autoref{fig:unseal-to-state}.
The wrapper required a manual proof, but the proof was short since the
wrapper is two lines of code.

Finally, to prove that {write} meets its confidentiality specification,
we similarly split {write} into a wrapper and a {write\_helper}.
The wrapper's job is to seal all input data and pass the handles to
{write\_helper}.  Much as with {read}, this reduced the proof effort
to just the wrapper.
