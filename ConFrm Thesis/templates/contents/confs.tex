\chapter{ConFs File System}

\section{Overview}
ConFs is a crash-safe file system implemented in Coq and its confidentiality is proven using ConFrm. It has a checksum based and encrypted log, a log cache for faster log reads, a transaction system for atomic operations, two block allocators for managing data and inode blocks, and an authentication system for access control. All these components are implemented on top of each other and intermediate abstraction layers are used to provide a cleaner interface for the components above.

\section{Implementation}

% Adopted from disksec
\section{Specifiying security}
% Explains special RDNI definitions.
Security of ConFs is defined as an RDNI specification for each of the compiled FileDisk operations. In the center of these specifications is equivalence relations between states. Since ConFs consists of four distinct levels, we had four different but related equivalence relations. We will explain these in the descending order.

\subsection{FileDisk Level Security}
We made some changes to the data noninterference and equivalence relation definitions to unify ret\_noninterference and state\_noninterference, and to make it general enough to ensure that it can accommodate special specifications for Write and ChangeOwner operations.

First change is that RNDI is defined over two programs to allow it to apply two writes with different input data. Second change is making return value equality conditional on the user. This change allow us to state "if the user is adversary, then return values should be the same." formally in the same definition.

\begin{figure}[ht]
    \centering
\begin{verbatim}
Definition RDNI (u: user) {T} (p1 p2: L_abs.(prog) T)
(rec: L_abs.(prog) unit) (valid_state: L_abs.(state) -> Prop)
(R: L_abs.(state) -> L_abs.(state) -> Prop) (cond: user -> Prop)
l_get_reboot_state :=
  forall lo s1 s1',
    L_abs.(recovery_exec) u lo s1 l_get_reboot_state p1 rec s1' ->
    valid_state s1 ->
    forall s2,
    valid_state s2 ->
    R s1 s2 ->
    exists s2',
      L_abs.(recovery_exec) u lo s2 l_get_reboot_state p2 rec s2' /\
      R (extract_state_r s1') (extract_state_r s2') /\
      (cond u -> extract_ret_r s1' = extract_ret_r s2') /\
      valid_state (extract_state_r s1') /\
      valid_state (extract_state_r s2').
\end{verbatim}
    \caption{Relatively Deterministic Noninterference (RDNI)}
    \label{fig:RDNI}
\end{figure}

We also modified our equivalence relation to take an optional inode number to exclude the contents of the file that its owner is being changed.
Since equivalence relation for ChangeOwner excludes contents of the file that is being operated on, it only provides half of the required security - i.e. it excludes leakage from the changed file to the outside. The fact that no information leaks from outside to the file is covered by its functional correctness which states that file's contents stay unchanged after the operation.

\begin{figure}[ht]
    \centering
\begin{verbatim}
Definition same_for_user_except (u: user) 
(exclude: option addr) (d1 d2: FD.(state)) :=
  addrs_match_exactly d1 d2 /\
  (forall inum file1 file2,
     exclude <> Some inum ->
     d1 inum = Some file1 ->
     d2 inum = Some file2 ->
     (file1.(owner) = u \/
      file2.(owner) = u) ->
     file1 = file2) /\
  (forall inum file1 file2,
     d1 inum = Some file1 ->
     d2 inum = Some file2 ->
     file1.(owner) = file2.(owner) /\ 
     length file1.(blocks) = length file2.(blocks)).
\end{verbatim}
    \caption{Equivalence relation for two FileDisk states.}
    \label{fig:eqivalence_for_filedisk}
\end{figure}



\subsection{AuthenticatedDisk and TransactionCache Level Security}
ConFrm provides a function that converts an equivalence over abstract states to an equivalence over implementation states given that a refinement between two exists. Informally, function states that two implementation states are equivalent if there exists two equivalent abstract states that are refined by the implementation states.

\begin{figure}[ht]
    \centering
\begin{verbatim}
Definition refines_related 
 (related_abs:  L_abs.(state) -> L_abs.(state) -> Prop)
 (si1 si2: L_imp.(state)) : Prop :=
exists (sa1 sa2: L_abs.(state)),
  R.(refines) si1 sa1 /\
  R.(refines) si2 sa2 /\
  related_abs sa1 sa2.
\end{verbatim}
    \caption{Equivalence conversion function from ConFrm}
    \label{fig:refines_related}
\end{figure}

This function alone was sufficient derive the equivalence for these two levels. However, this is not generally is the case. Sometimes, extra conditions are needed to establish the relation between the parts of the implementation state that is abstracted away. We will present examples of such instances in the following section.

Main reason for this is that, oracles in ConFrm implicitly dictates number of execution steps a program takes. Semantics of a program requires consumption of an exactly one token per operation executed. This requirement generally implies two noninterfering programs have to follow the same execution paths. 

%%%% ADD an example noninterference spec here?

\subsection{CachedDisk Level Security}
{\color{red} INTRO HERE}

Consider the following read function:

\begin{figure}[ht]
    \centering
\begin{verbatim}
Definition read a :=
  mv <- Cache_Read a;
  if mv = Some v then 
    Ret v
  else
    Disk_Read a
\end{verbatim}
    \label{fig:refines_related}
\end{figure}
%
Proving noninterference of this function requires showing that if there is an execution of the function from a state with a particular oracle, then there must be an execution from a related state with the same oracle. Since having same oracle dictates that programs have to follow the same execution paths, two related states have to contain exactly same addresses in their caches (corresponding data could be different).

If an abstraction of the states hides the existence of a cache, then refining two related abstract states is not sufficient to capture this requirement. In this instance, equivalence relation needs to be supplemented with this property to make it finer-grained.

This particular example and some other more complicated variants are present in log functions. Therefore, we supplemented the equivalence relation with the following:

{\color{red} Property here}

{\color{red} Explanation of the property here}


\section{Proving security}
We proved confidentiality of ConFs with two general categories of theorems: (1) confidentiality proofs of FileDisk operations, and (2) premises required to establish confidentiality of the refining programs of the operations. Proving (1) was relatively easy thanks to the simple structure of FileDisk state and semantics. Majority of the effort among them went into proving (2).

Proving (2) required us to prove 2 different properties for each operation: (a) oracle refinement being independent of confidential data, and (b) existence of a refined abstract oracle. Proofs in a compositional manner - i.e. corresponding properties are proved for each operation of each layer, then brought together to construct required proofs. Compositional nature of the properties significantly reduced the proof effort and repetition between layers.

One interesting case appeared regarding write operations that overwrite some data with itself. Possibility of such operations made it impossible to determine if a write succeeded or not after a crash by just examining the disk's final state. To resolve this problem, we had to reason about precise number of steps the function ran as well as the required conditions on the crash and reboot states of the disk after the execution.  This precise and low level reasoning was tedious and required significant proof effort. Despite our best efforts, question of whether this can be solved at meta-theoretic level or with another proof strategy that requires less effort remains open.

During this reasoning, we had to employ the condition that, in the log, current and residual blocks at the same address can't be identical. Encryption of each block ensured that such a condition is exponentially unlikely. By design, since a residual block and a current block cannot belong to the same transaction, they must be encrypted with different keys. Therefore such a condition does not weaken the conclusion significantly.


{\color{red} Termination Sensitivity may go here. Maybe an explanation of why it was as hard as it is?}
