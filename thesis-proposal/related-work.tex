\section{Related Work}

ConFrm and ConFs build on a diverse body of prior work. We will explain these works throughout this section. 

\paragraph{Noninterference properties.} There is a significant body of work about formalizing noninterference properties [19, 25, 26, 29, 30, 32, disksec]. ConFrm’s relatively deterministic noninterference builds upon this existing work. 

ConFrm’s definition is different from its predecessors in how it treats nondeterminism in its formalism. ConFrm takes a more fine-grained approach in relating nondeterministic executions by requiring a strong coupling between executions for each nondeterministic execution branch. 

%rewrite this for paper
\paragraph{Machine-checked security in systems.} 
Several prior projects have proven security (and specifically confidentiality) properties about their system implementations: seL4 [23, 26], CertiKOS [15], Ironclad [20] and DiskSec [??]. For seL4 and CertiKOS, the theorems prove complete isolation: CertiKOS requires disabling IPC to prove its security theorems, and seL4’s security theorem requires disjoint sets of capabilities. In the context of a file system, complete isolation is not possible: one of the main goals of a file system is to enable sharing. Furthermore, CertiKOS is limited to proving security via deterministic specifications. Nondeterminism is important in a file system to handle crashes and to abstract away implementation details in specifications. 


Ironclad proves that several applications, such as a notary service and a password-hashing application, do not disclose their own secrets (e.g., a private key), formulated as noninterference. Also using noninterference, Komodo [17] reasons about confidential data in an enclave and shows that an adversary cannot learn the confidential data. Ironclad and Komodo’s approach cannot specify or prove a file system: both systems have no notion of a calling principal or support for multiple users, and there is no possibility of returning confidential data to some principals (but not others). Finally, there is no support for nondeterministic crashes.


DiskSec supports nondeterministic crashes, discretionary access control, and shared data structures. However, it lacks support for branching on confidential data (e.g. hash-based logging), abstraction layers, and provides weaker guarantees for crashes.

%rewrite this for paper
\paragraph{Information flow and type systems.} 
Another approach to ensuring confidentiality involves relying on type systems. An advantage of this approach is that type checking can be automated to reduce proof load for the developer. 


Type systems and static-analysis algorithms, as with Jif’s labels [27, 28] and the UrFlow analysis [14], have been developed to reason about information-flow properties of application code. However, these analyzers are static and would be hard to use for reasoning about data structures inside of a file system (such as a write-ahead log or a buffer cache) that contain data from different users.

SeLoc[??] uses double weakest preconditions to prove noninterference for fine-grained concurrent programs. It is built on top of IRIS[??], a separation logic-based framework that proves correctness of fine-grained concurrent programs. It provides a confidentiality-ensuring type system and a wide array of tools to the developers. However, employing SeLoc requires using IRIS, which adds a substantial entry barrier. Conversely, ConFrm is standalone and lightweight, but does not offer the full array of tools SeLoc offers. 

\paragraph{Sequential composability and confidentiality preserving refinements.} 
Since it is known that noninterference is not preserved in the general case[??], there is a body of work that tries to identify the conditions that make refinements noninterference preserving. 

 
Sun et. al. [??] proposes two confidentiality properties for interface automata: SIR-GNNI and RRNI. Both properties are based on refinements and defined relative to arbitrary security lattices. They also provide sufficient conditions that make SIR-GNNI and RRNI sequentially compositional. 

Baumann et. al. [??] formulates noninterference as an epistemic logic over trace sets. They define ignorance preserving refinements and prove that it is a sufficient condition to preserve the noninterference of abstraction. They also show that ignorance preserving refinements are not compositional w.r.t. sequential composition. They propose another class of refinements called “relational refinements” , which are sequentially compositional. 


ConFrm’s definition is also sequentially compositional. Our suggested property differs from existing work in two points. First, it provides stronger guarantees for nondeterministic executions. Second, it supports reasoning about crashes and recovery of the storage system. 


	%https://www.google.com/books/edition/Principles_of_Security_and_Trust/yKlVDwAAQBAJ?hl=en&gbpv=1&dq=noninterference+composition&pg=PA53&printsec=frontcover
	
	%https://arxiv.org/abs/1910.00905
	
	%https://library.oapen.org/bitstream/handle/20.500.12657/27744/1002261.pdf?sequence=1#page=65
	
	%Sun
	%https://ieeexplore.ieee.org/abstract/document/8029662?casa_token=dQhnCKWA07UAAAAA:AEf00GwLFCEQH78YmZVLyT0ny1-DSYr6g4ibsJiq83pLMpBms7xetLCBh81G9oa0B3_4-y4
	
	%Baumann
	%https://publications.cispa.saarland/3213/1/InfoFlowRefinement.pdf
