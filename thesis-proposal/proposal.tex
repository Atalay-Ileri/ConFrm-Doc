\documentclass[onecolumn]{paper}
\usepackage{tabularx}
\usepackage{color}
\usepackage{listings}
\usepackage{verbatim}
\usepackage{quiver}
\usepackage{caption}
\usepackage{subcaption}

\newcolumntype{L}{>{\raggedright\arraybackslash}X}

\title{ConFrm: Confidentiality Framework for Crash Safe Storage Systems}
\author{Atalay Mert Ileri \and Frans Kaashoek \and Nickolai Zeldovich}

\begin{document}
\maketitle
%%%%%%%%%%%%%
% TODO
%%%%%%%

\section{Introduction}
As computation and storage paradigms shift toward the cloud, the safety of computation and data is becoming increasingly important. Users expect their data stored in such systems to stay secret against malicious third parties. Precise descriptions of how such a system should behave to ensure confidentiality of the stored data are called confidentiality specifications. A confidentiality specification of a system both forces a system to maintain certain properties, and informs users about the safety guarantees that system provides. 

Writing a confidentiality specification is generally more difficult than writing a correctness specification. To demonstrate the possible hardships of writing a specification that can prevent malicious implementations, we can look at a specification for the create() operation from our confidential file system ConFs. create() takes an owner and creates an empty file owned by the provided owner. Upon successful completion, it returns the inode number that points to that file. A natural correctness specification for its return value could be “create() returns the index of the previously unused inode that now corresponds to the newly created file.” If we were only interested in functional correctness, this could be an acceptable specification. However, there is a substantial confidentiality problem associated with it. create() is allowed to return any inode number as long as it was unused at the time of the call. A malicious implementation can take advantage of this nondeterminism in the specification to pick the returned inode number and subsequently leak confidential data (e.g. last byte of the return value being equal to the first byte of a block that belongs to another user).

This problem, and others of the same nature, could be solved by writing a fully deterministic specification for create() that exactly pinpoints what it should return and what the file-system state should be after its completion. However, such a specification would be overly verbose, obscuring the important parts of the specification, making it difficult to read and comprehend, and essentially defeating one of its key purposes. Therefore, a good confidentiality specification should address such vulnerabilities in its scope, but should also be clear and concise so it can be read, understood and reviewed by humans. Finding the right balance between these conflicting requirements is what makes writing a confidentiality specification challenging.

Our proposed solution to the above problems is ConFrm, a framework for 
implementing and proving the confidentiality of storage systems in a modular fashion. ConFrm provides a noninterference definition with better guarantees for nondeterministic behavior, as well as the required tools and necessary conditions for safety-preserving abstractions. These two components of ConFrm enable us to overcome the limitations of previous works and provide a simple yet powerful tool for implementing safe storage systems.


To test the capabilities of ConFrm, we implemented ConFS, a confidential file system that is crash-resistant via checksum logging, transactional operations, and coarse-grained discretionary access control. 

Both ConFrm and ConFS are implemented in Coq. All proofs are fully machine-checked to ensure their correctness. 


%%%%%%%%%%%%%
% TODO
%%%%%%%

\section{Problems and Motivation}
\subsection{Unsafety from Probability}
We are trying to tackle two problems in this work. The first issue is the interaction between nondeterminism and noninterference. More specifically, standard noninterference definitions fail to address leakages that result from different frequencies of possible execution traces of the same program. A simple example of this weakness can be seen below: 

\begin{lstlisting}
if (random_bit() = 1)
  return secret_bit
else
  return random_bit()
\end{lstlisting}

This code leaks the secret bit 50\% of the time and outputs a random bit 50\% of the time. It is also important to note that it can output 0 or 1 independently of the secret value. This way, any (state, return) pair represents a successful execution. 

Let's say that two states are equivalent if they are the same for their non-secret parts. Since any pair of state and return value is a valid execution, it is possible to find another execution from a related state. This makes it satisfy the noninterference definition.

Although the above program satisfies the noninterference definition, the secret bit can be determined via the observed frequency of repeated calls. When we look at the frequencies of output values, we can see that they correlate with the value of the secret bit.\\

\begin{tabular}{| c | c | c |}
	\hline
	Secret bit & Output 0 \% & Output 1 \% \\
	\hline
	0 &	75\% & 25\% \\
	\hline
	1 &	25\% & 75\% \\
	\hline
\end{tabular}\\

Any adversary who can observe the output of the function sufficiently many times can infer the value of the secret bit with high confidence. These types of vulnerabilities are not limited to usage of randomization. They also manifest themselves when there are random events that can affect the behavior of the system. 

The main example we will consider is crash and recovery of a storage system. If the recovery behavior of the system is dependent on the secret data stored in it, then the system may contain a vulnerability of this kind. 

\subsection{Unsafe Data Handling}
The second problem observed involves handling the data in a potentially unsafe way. Many storage systems process the data provided to them before storing it. A potentially malicious developer can take advantage of this capability to leak information.

A classic example of this phenomenon is branching on confidential data. If not done carefully, it can result in leakage of confidential data. This does not mean that branching on confidential data should be avoided at all costs.

Data deduplication in a storage system in a good example to demonstrate this effect. For this example, we will treat data contents as secret but metadata as public. Therefore a user noticing extra space savings is fine as long as he cannot infer contents of the data from this knowledge.

Deduplication requires branching on confidential data, which can be abused if not handled carefully. Consider the following obviously unsafe implementation:

\begin{lstlisting}
write_to_txn(v)
  if (v is in txn)
    return false
  else
    add_to_txn(v)
    return true
\end{lstlisting}

This implementation allows an adversary to query the contents of the transaction. 
A simple solution that can be considered to solve this problem is prohibiting branching control flow based on confidential data. This can be achieved by using a deeply embedded language that doesn't have an operation to compare secret data or treating secret data as an abstract object that can't be compared. Such approaches have the advantage 
that any program written in such a language is noninterfering by construction. 

It is quite a strong property but also a restrictive one which eliminates many possible efficient implementations.

However, it is possible to implement a safe deduplicator. One way would be performing the process after the transaction is finalized but before committing it. Such a deferred deduplication will still branch on confidential data but without revealing it.

An ideal solution would prohibit unsafe implementations while allowing the safe ones with as little constraint as possible.  

%%%%%%%%%%%%%
% TODO
%%%%%%%

\section{Solution Approach}
To address the problems stated above, we used 3 approaches:
\begin{itemize}
	\item Quantification of nondeterminism via tokens to solve unsafety from probability.
	\item Abstraction via layer system to solve unsafe data handling.
	\item Property transfer via simulation proofs to enable usage of layers while maintaining safety.
\end{itemize}


\subsection{Quantification of Nondeterminism}
To be able to address unsafety coming from certain kinds of probabilistic behavior, we decided to quantize the nondeterminism. We enhanced our execution semantics with a sequence, such that, whenever a nondeterministic choice needs to be made during execution, the sequence will contain the option that will be chosen. For all possible execution traces a program have under general nondeterministic semantics, there should be a sequence that will guide the execution to that result. 

Our assumption is that nondeterminism in the system is independent of the contents of secret data but can depend on the public metadata. For example, size of the data can affect which nondeterministic choices are available but actual data cannot. This assumption is enforced as a proof obligation to use our technique.

This approach have two benefits. First and most significant, it enables us to
turn nondeterministic executions into relatively deterministic ones. 
Relatively deterministic execution is defined as an execution trace that is 
deterministic given a particular sequence of nondeterministic choices.

A simple example of this is execution of an operation that generates an encryption key. 
A nondeterministic execution rule would be

$$\forall\ s\ key,\ s  =GenerateKey\Rightarrow (s, key)$$ 

This rule states that when we run GenerateKey we can get any key as a result of the execution.

A relatively deterministic execution rule would be

$$\forall\ s\ key,\ s  =[key]GenerateKey\Rightarrow (s, key)$$ 

Difference between this rule and the previous one is subtle but crucial. 
According to this rule, there is only one possible key you can get when you execute GenerateKey in this specific conditions.

In some sense, it can be thought as which key you will get is predetermined by the state of the world. Program execution is deterministic relative to the state of the world.
In contrast, you could get any key in the nondeterministic execution regardless of the state of the world.

With this modified definition, we managed to state and prove a stronger version of noninterference: relatively deterministic noninterference (RDNI). RDNI requires that, for any sequence of nondeterministic choices, if there is a completed execution of a program from a state, then there exists another execution of a related program from a related state with same choice sequence.

\begin{figure}[H]
	$\forall\ o\ s_1\ s_2\ s_1'\ v,$\\
	$s_1\approx s_2 \rightarrow$\\
	$s_1 =[o]p_1\Rightarrow (s_1', v) \rightarrow$\\
	$\exists s_2',\ s_2 =[o]p_2\Rightarrow (s_2', v)\ \wedge\ s_1'\approx s_2'.$\\
	\caption{Relatively Deterministic Noninterference}
\end{figure}

RDNI's requirement for matching execution from any choice sequence ensures that no matter which nondeterministic branch is followed during the execution, there will be an indistinguishable execution from an equivalent state. This one-to-one correspondence implies that a state cannot be distinguished based on repeated observation of the system.

If we revisit random bit example, we can show that it is indeed does not satisfy RDNI. For the choice sequence that chooses first random bit as 1, there is no indistinguishable execution between the state where secret bit is 0 and the state where secret bit is 1.

We also modified termination sensitivity and simulation definitions to integrate relative determinism. Modification of termination sensitivity was straightforward: executions are parameterized by the same oracle. Modifying simulation definition required a more complicated approach. We extended traditional refinements with oracle refinement relation.
Oracle refinement relation is parameterized by a user, an implementation state and an abstract program and will be denoted with $R_o$. We omit the user, state and program parameters in notation when it is clear from the context. Our modified relation is

\begin{figure}[H]
$\forall\ o_i\ o_a\ i\ i'\ a\ v,$\\
$i\ R\ a \rightarrow$\\
$o_i\ R_o\ o_a \rightarrow$\\
$i =[o_i] R(p)\Rightarrow (i', v) \rightarrow$\\
$\exists a',\ a =[o_r] p\Rightarrow (a', v)\ \wedge\ i'\ R\ a'.$\\
	\caption{Simulation with Oracles}
\end{figure}

\subsection{Abstraction via layer system}
Abstraction is a useful tool when dealing with large and complex software systems.
It allows developers to hide implementation details of different modules of a software and purely think in terms of its observed behavior.

% AI - Rewrite or delete this
Our layer system streamlines the abstraction of implementations by providing tools that makes defining a new DSL that exhibits the same behavior with the implementation easier.
This is crucial in hiding the details of functions that use potentially unsafe primitives in a noninterfering fashion.

Once an implementation is abstracted away with a layer, it becomes completely opaque to the rest of the system. Our system guarantees that an implementation that simulates a noninterfering layer is also noninterfering.

Because the operations and state of a layer description are simpler than its underlying implementation, proving their noninterference becomes less complex as well. Additionally, implementations under the layer can be changed without affecting other proofs in the system.

Transforming the representation of a system state has the benefit of simplifying how system is modeled. For example, a file system can be abstracted as a partial mapping between file names to file contents. Once properly abstracted, all the underliying complexities (e.g. inodes, log, block allocators etc.) becomes irrelevant to the system's behavior. This allows developers to define and prove simple confidentiality statements over simple representation. For instance, assigning an owner to each file is much more straightforward than designating each inode and data blocks to an owner, and subsequently keeping track of the information through varying stages of operations. This technique may even be impossible if implementation does optimizations that fit multiple owners' data in a single data unit (e.g. a disk block).

\subsection{NI Transfer via Simulation Proofs}
It is known that, in general nondeterministic case, NI is not preserved under simulation, but under bisimulation. This creates a problem for abstraction techniques, which relies on simulation proofs to show the behavior of implementation is a subset of the abstraction's. Since bisimulation is a stricter requirement than simulation, the amount of details an abstraction can omit is significantly reduced. Furthermore, you cannot introduce any new behavior or remove any existing behavior in an abstraction.

This restriction eliminated one of the best advantages of abstraction techniques, namely simplifying a system's model.
On top of that, bisimulation proofs are much more cumbersome compared to simulation proofs. 

However we show that RDNI is preserved under simulations combined with some auxiliary properties. This relieves proof burden of the developer significantly as well as increases power of abstraction layers. Our layer system makes it possible to construct complex systems as isolated, self contained pieces; just like normal software systems.

Our core theorem states that, if an abstract layer is RDNI w.r.t. $\approx$, then an implementation layer that refines the abstract layer is RDNI w.r.t. $R(\approx)$.  

Crux of the technique is unifying two abstract executions via relative determinism. First execution is obtained by simulation followed by noninteference of the abstract layer. Second execution is obtained by simulation from related state. Relative determinism ensures that these two executions and their final states are one and the same. Once this fact is established, final states of the two implementation executions become equivalent by definition.

To be able to employ the theorem in termination-sensitive way, one needs to show that $R(\approx)$ is termination sensitive w.r.t. programs p1 and p2 we are reasoning about.

%\begin{figure}
	\centering 
% https://q.uiver.app/?q=WzAsOCxbMCwxLCJpIl0sWzIsMSwiaSciXSxbMCwwLCJhIl0sWzQsMSwiaSJdLFs2LDEsImknIl0sWzQsMCwiYSJdLFs2LDAsImEnIl0sWzIsMF0sWzAsMSwiUihwKSIsMV0sWzAsMiwiUiIsMV0sWzMsNSwiUiIsMV0sWzMsNCwiUihwKSIsMV0sWzQsNiwiUiIsMV0sWzUsNiwicCIsMV0sWzEsNywiIiwxLHsic3R5bGUiOnsiYm9keSI6eyJuYW1lIjoibm9uZSJ9LCJoZWFkIjp7Im5hbWUiOiJub25lIn19fV0sWzE0LDEwLCIiLDEseyJzaG9ydGVuIjp7InNvdXJjZSI6MzAsInRhcmdldCI6MzB9fV1d
\begin{tikzcd}
	a && {} && a && {a'} \\
	i && {i'} && i && {i'}
	\arrow["{R(p)}"{description}, from=2-1, to=2-3]
	\arrow["R"{description}, from=2-1, to=1-1]
	\arrow[""{name=0, anchor=center, inner sep=0}, "R"{description}, from=2-5, to=1-5]
	\arrow["{R(p)}"{description}, from=2-5, to=2-7]
	\arrow["R"{description}, from=2-7, to=1-7]
	\arrow["p"{description}, from=1-5, to=1-7]
	\arrow[""{name=1, anchor=center, inner sep=0}, draw=none, from=2-3, to=1-3]
	\arrow[shorten <=19pt, shorten >=19pt, Rightarrow, from=1, to=0]
\end{tikzcd}

\caption{Simulation}

% https://q.uiver.app/?q=WzAsOCxbMCwxLCJpXzIiXSxbNCwwLCJpXzEiXSxbNiwwLCJpXzEnIl0sWzAsMCwiaV8xIl0sWzIsMCwiaV8xJyJdLFs0LDEsImlfMiJdLFs2LDEsImlfMiciXSxbMywwXSxbMSwyLCJSKHApIiwxXSxbMywwLCJSKEUpIiwxLHsibGV2ZWwiOjIsInN0eWxlIjp7ImhlYWQiOnsibmFtZSI6Im5vbmUifX19XSxbMyw0LCJSKHApIiwxXSxbMSw1LCJSKEUpIiwxLHsibGV2ZWwiOjIsInN0eWxlIjp7ImhlYWQiOnsibmFtZSI6Im5vbmUifX19XSxbNSw2LCJSKHApIiwxXSxbMCw3LCIiLDEseyJzdHlsZSI6eyJib2R5Ijp7Im5hbWUiOiJub25lIn0sImhlYWQiOnsibmFtZSI6Im5vbmUifX19XSxbMTMsMTEsIiIsMSx7InNob3J0ZW4iOnsic291cmNlIjo0MCwidGFyZ2V0IjoyMH19XV0=
\begin{tikzcd}
	{i_1} && {i_1'} & {} & {i_1} && {i_1'} \\
	{i_2} &&&& {i_2} && {i_2'}
	\arrow["{R(p)}"{description}, from=1-5, to=1-7]
	\arrow["{R(E)}"{description}, Rightarrow, no head, from=1-1, to=2-1]
	\arrow["{R(p)}"{description}, from=1-1, to=1-3]
	\arrow[""{name=0, anchor=center, inner sep=0}, "{R(E)}"{description}, Rightarrow, no head, from=1-5, to=2-5]
	\arrow["{R(p)}"{description}, from=2-5, to=2-7]
	\arrow[""{name=1, anchor=center, inner sep=0}, draw=none, from=2-1, to=1-4]
	\arrow[shorten <=32pt, shorten >=16pt, Rightarrow, from=1, to=0]
\end{tikzcd}

\caption{Termination Sensitivity}
% https://q.uiver.app/?q=WzAsOCxbMCwwLCJhXzEiXSxbMCwxLCJhXzIiXSxbMywwLCJhXzEiXSxbNCwwLCJhXzEnIl0sWzMsMSwiYV8yIl0sWzEsMCwiYV8xJyJdLFs0LDEsImFfMiciXSxbMiwwXSxbMCwxLCJFIiwxLHsibGV2ZWwiOjIsInN0eWxlIjp7ImhlYWQiOnsibmFtZSI6Im5vbmUifX19XSxbMCw1LCJwIiwxXSxbMiwzLCJwIiwxXSxbMiw0LCJFIiwxLHsibGV2ZWwiOjIsInN0eWxlIjp7ImhlYWQiOnsibmFtZSI6Im5vbmUifX19XSxbNCw2LCJwIiwxXSxbMyw2LCJFIiwxLHsibGV2ZWwiOjIsInN0eWxlIjp7ImhlYWQiOnsibmFtZSI6Im5vbmUifX19XSxbMSw3LCIiLDEseyJzdHlsZSI6eyJib2R5Ijp7Im5hbWUiOiJub25lIn0sImhlYWQiOnsibmFtZSI6Im5vbmUifX19XSxbMTQsMTEsIiIsMSx7InNob3J0ZW4iOnsic291cmNlIjozMCwidGFyZ2V0IjozMH19XV0=
\begin{tikzcd}
	{a_1} & {a_1'} & {} & {a_1} & {a_1'} \\
	{a_2} &&& {a_2} & {a_2'}
	\arrow["E"{description}, Rightarrow, no head, from=1-1, to=2-1]
	\arrow["p"{description}, from=1-1, to=1-2]
	\arrow["p"{description}, from=1-4, to=1-5]
	\arrow[""{name=0, anchor=center, inner sep=0}, "E"{description}, Rightarrow, no head, from=1-4, to=2-4]
	\arrow["p"{description}, from=2-4, to=2-5]
	\arrow["E"{description}, Rightarrow, no head, from=1-5, to=2-5]
	\arrow[""{name=1, anchor=center, inner sep=0}, draw=none, from=2-1, to=1-3]
	\arrow[shorten <=19pt, shorten >=19pt, Rightarrow, from=1, to=0]
\end{tikzcd}

\caption{RDNI}
\end{figure}

%This is how our version proves transfer
\subsection*{Illustration of transfer theorem}

a) Initial state
% https://q.uiver.app/?q=WzAsNSxbMCwzLCJpXzEiXSxbMiwxLCJpXzEnIl0sWzAsMCwiYV8xIl0sWzQsMCwiYV8yIl0sWzQsMywiaV8yIl0sWzAsMSwiIiwxLHsic3R5bGUiOnsiYm9keSI6eyJuYW1lIjoiZGFzaGVkIn19fV0sWzAsMiwiIiwxLHsic3R5bGUiOnsidGFpbCI6eyJuYW1lIjoiYXJyb3doZWFkIn19fV0sWzQsMywiIiwxLHsic3R5bGUiOnsidGFpbCI6eyJuYW1lIjoiYXJyb3doZWFkIn19fV0sWzAsNCwiIiwxLHsic3R5bGUiOnsidGFpbCI6eyJuYW1lIjoiYXJyb3doZWFkIn19fV0sWzIsMywiIiwxLHsic3R5bGUiOnsidGFpbCI6eyJuYW1lIjoiYXJyb3doZWFkIn19fV1d
\[\begin{tikzcd}
	{a_1} &&&& {a_2} \\
	&& {i_1'} \\
	\\
	{i_1} &&&& {i_2}
	\arrow[dashed, from=4-1, to=2-3]
	\arrow[tail reversed, from=4-1, to=1-1]
	\arrow[tail reversed, from=4-5, to=1-5]
	\arrow[tail reversed, from=4-1, to=4-5]
	\arrow[tail reversed, from=1-1, to=1-5]
\end{tikzcd}\]

b) Use simulation
% https://q.uiver.app/?q=WzAsNyxbMCw1LCJpXzEiXSxbMiwzLCJpXzEnIl0sWzAsMiwiYV8xIl0sWzQsMiwiYV8yIl0sWzQsNSwiaV8yIl0sWzUsMywiaV8yJyJdLFsyLDAsImFfMSciXSxbMCwxLCIiLDEseyJzdHlsZSI6eyJib2R5Ijp7Im5hbWUiOiJkYXNoZWQifX19XSxbMCwyLCIiLDEseyJzdHlsZSI6eyJ0YWlsIjp7Im5hbWUiOiJhcnJvd2hlYWQifX19XSxbNCwzLCIiLDEseyJzdHlsZSI6eyJ0YWlsIjp7Im5hbWUiOiJhcnJvd2hlYWQifX19XSxbMCw0LCIiLDEseyJzdHlsZSI6eyJ0YWlsIjp7Im5hbWUiOiJhcnJvd2hlYWQifX19XSxbMiwzLCIiLDEseyJzdHlsZSI6eyJ0YWlsIjp7Im5hbWUiOiJhcnJvd2hlYWQifX19XSxbNCw1XSxbMiw2XSxbNiwxLCIiLDEseyJzdHlsZSI6eyJ0YWlsIjp7Im5hbWUiOiJhcnJvd2hlYWQifX19XV0=
\[\begin{tikzcd}
	&& {a_1'} \\
	\\
	{a_1} &&&& {a_2} \\
	&& {i_1'} &&& {i_2'} \\
	\\
	{i_1} &&&& {i_2}
	\arrow[dashed, from=6-1, to=4-3]
	\arrow[tail reversed, from=6-1, to=3-1]
	\arrow[tail reversed, from=6-5, to=3-5]
	\arrow[tail reversed, from=6-1, to=6-5]
	\arrow[tail reversed, from=3-1, to=3-5]
	\arrow[from=6-5, to=4-6]
	\arrow[from=3-1, to=1-3]
	\arrow[tail reversed, from=1-3, to=4-3]
\end{tikzcd}\]

c) Use abstract layer RDNI
% https://q.uiver.app/?q=WzAsOCxbMCw1LCJpXzEiXSxbMiwzLCJpXzEnIl0sWzAsMiwiYV8xIl0sWzQsMiwiYV8yIl0sWzQsNSwiaV8yIl0sWzUsMywiaV8yJyJdLFsyLDAsImFfMSciXSxbNSwwLCJhXzInIl0sWzAsMSwiIiwxLHsic3R5bGUiOnsiYm9keSI6eyJuYW1lIjoiZGFzaGVkIn19fV0sWzAsMiwiIiwxLHsic3R5bGUiOnsidGFpbCI6eyJuYW1lIjoiYXJyb3doZWFkIn19fV0sWzQsMywiIiwxLHsic3R5bGUiOnsidGFpbCI6eyJuYW1lIjoiYXJyb3doZWFkIn19fV0sWzAsNCwiIiwxLHsic3R5bGUiOnsidGFpbCI6eyJuYW1lIjoiYXJyb3doZWFkIn19fV0sWzIsMywiIiwxLHsic3R5bGUiOnsidGFpbCI6eyJuYW1lIjoiYXJyb3doZWFkIn19fV0sWzQsNV0sWzIsNl0sWzYsMSwiIiwxLHsic3R5bGUiOnsidGFpbCI6eyJuYW1lIjoiYXJyb3doZWFkIn19fV0sWzMsN10sWzYsNywiIiwxLHsic3R5bGUiOnsidGFpbCI6eyJuYW1lIjoiYXJyb3doZWFkIn19fV1d
\[\begin{tikzcd}
	&& {a_1'} &&& {a_2'} \\
	\\
	{a_1} &&&& {a_2} \\
	&& {i_1'} &&& {i_2'} \\
	\\
	{i_1} &&&& {i_2}
	\arrow[dashed, from=6-1, to=4-3]
	\arrow[tail reversed, from=6-1, to=3-1]
	\arrow[tail reversed, from=6-5, to=3-5]
	\arrow[tail reversed, from=6-1, to=6-5]
	\arrow[tail reversed, from=3-1, to=3-5]
	\arrow[from=6-5, to=4-6]
	\arrow[from=3-1, to=1-3]
	\arrow[tail reversed, from=1-3, to=4-3]
	\arrow[from=3-5, to=1-6]
	\arrow[tail reversed, from=1-3, to=1-6]
\end{tikzcd}\]


d) Use simulation and relative determinism
% https://q.uiver.app/?q=WzAsMTAsWzAsNSwiaV8xIl0sWzIsMywiaV8xJyJdLFswLDIsImFfMSJdLFs0LDIsImFfMiJdLFs0LDUsImlfMiJdLFs1LDNdLFsyLDAsImFfMSciXSxbNSwwLCJhXzInIl0sWzYsMCwiYV8yJyciXSxbNiwzLCJpXzInIl0sWzAsMSwiIiwxLHsic3R5bGUiOnsiYm9keSI6eyJuYW1lIjoiZGFzaGVkIn19fV0sWzAsMiwiIiwxLHsic3R5bGUiOnsidGFpbCI6eyJuYW1lIjoiYXJyb3doZWFkIn19fV0sWzQsMywiIiwxLHsic3R5bGUiOnsidGFpbCI6eyJuYW1lIjoiYXJyb3doZWFkIn19fV0sWzAsNCwiIiwxLHsic3R5bGUiOnsidGFpbCI6eyJuYW1lIjoiYXJyb3doZWFkIn19fV0sWzIsMywiIiwxLHsic3R5bGUiOnsidGFpbCI6eyJuYW1lIjoiYXJyb3doZWFkIn19fV0sWzIsNl0sWzYsMSwiIiwxLHsic3R5bGUiOnsidGFpbCI6eyJuYW1lIjoiYXJyb3doZWFkIn19fV0sWzMsN10sWzYsNywiIiwxLHsic3R5bGUiOnsidGFpbCI6eyJuYW1lIjoiYXJyb3doZWFkIn19fV0sWzMsOF0sWzQsOV0sWzgsOSwiIiwxLHsic3R5bGUiOnsidGFpbCI6eyJuYW1lIjoiYXJyb3doZWFkIn19fV0sWzcsOCwiIiwxLHsic3R5bGUiOnsiaGVhZCI6eyJuYW1lIjoibm9uZSJ9fX1dLFsxNywxOSwiIiwxLHsic2hvcnRlbiI6eyJzb3VyY2UiOjIwLCJ0YXJnZXQiOjIwfSwic3R5bGUiOnsiaGVhZCI6eyJuYW1lIjoibm9uZSJ9fX1dXQ==
\[\begin{tikzcd}
	&& {a_1'} &&& {a_2'} & {a_2''} \\
	\\
	{a_1} &&&& {a_2} \\
	&& {i_1'} &&& {} & {i_2'} \\
	\\
	{i_1} &&&& {i_2}
	\arrow[dashed, from=6-1, to=4-3]
	\arrow[tail reversed, from=6-1, to=3-1]
	\arrow[tail reversed, from=6-5, to=3-5]
	\arrow[tail reversed, from=6-1, to=6-5]
	\arrow[tail reversed, from=3-1, to=3-5]
	\arrow[from=3-1, to=1-3]
	\arrow[tail reversed, from=1-3, to=4-3]
	\arrow[""{name=0, anchor=center, inner sep=0}, from=3-5, to=1-6]
	\arrow[tail reversed, from=1-3, to=1-6]
	\arrow[""{name=1, anchor=center, inner sep=0}, from=3-5, to=1-7]
	\arrow[from=6-5, to=4-7]
	\arrow[tail reversed, from=1-7, to=4-7]
	\arrow[Rightarrow, no head, from=1-6, to=1-7]
	\arrow[shorten <=3pt, shorten >=3pt, Rightarrow, no head, from=0, to=1]
\end{tikzcd}\]


% https://q.uiver.app/?q=WzAsOCxbMCw1LCJpXzEiXSxbMiwzLCJpXzEnIl0sWzAsMiwiYV8xIl0sWzQsMiwiYV8yIl0sWzQsNSwiaV8yIl0sWzIsMCwiYV8xJyJdLFs2LDAsImFfMiciXSxbNiwzLCJpXzInIl0sWzAsMV0sWzAsMiwiIiwxLHsic3R5bGUiOnsidGFpbCI6eyJuYW1lIjoiYXJyb3doZWFkIn19fV0sWzQsMywiIiwxLHsic3R5bGUiOnsidGFpbCI6eyJuYW1lIjoiYXJyb3doZWFkIn19fV0sWzAsNCwiIiwxLHsic3R5bGUiOnsidGFpbCI6eyJuYW1lIjoiYXJyb3doZWFkIn19fV0sWzIsMywiIiwxLHsic3R5bGUiOnsidGFpbCI6eyJuYW1lIjoiYXJyb3doZWFkIn19fV0sWzIsNV0sWzUsMSwiIiwxLHsic3R5bGUiOnsidGFpbCI6eyJuYW1lIjoiYXJyb3doZWFkIn19fV0sWzUsNiwiIiwxLHsic3R5bGUiOnsidGFpbCI6eyJuYW1lIjoiYXJyb3doZWFkIn19fV0sWzMsNl0sWzQsN10sWzYsNywiIiwxLHsic3R5bGUiOnsidGFpbCI6eyJuYW1lIjoiYXJyb3doZWFkIn19fV0sWzEsNywiIiwxLHsic3R5bGUiOnsidGFpbCI6eyJuYW1lIjoiYXJyb3doZWFkIn19fV1d
e) By definition of E'
\[\begin{tikzcd}
	&& {a_1'} &&&& {a_2'} \\
	\\
	{a_1} &&&& {a_2} \\
	&& {i_1'} &&&& {i_2'} \\
	\\
	{i_1} &&&& {i_2}
	\arrow[from=6-1, to=4-3]
	\arrow[tail reversed, from=6-1, to=3-1]
	\arrow[tail reversed, from=6-5, to=3-5]
	\arrow[tail reversed, from=6-1, to=6-5]
	\arrow[tail reversed, from=3-1, to=3-5]
	\arrow[from=3-1, to=1-3]
	\arrow[tail reversed, from=1-3, to=4-3]
	\arrow[tail reversed, from=1-3, to=1-7]
	\arrow[from=3-5, to=1-7]
	\arrow[from=6-5, to=4-7]
	\arrow[tail reversed, from=1-7, to=4-7]
	\arrow[tail reversed, from=4-3, to=4-7]
\end{tikzcd}\]
\end{document}

\begin{comment}

\subsection*{System we consider}
We are consider a storage system with a course-grained ownership. smallest data unit will be referred as a data block, Data are accessed with "a handle", which represents a group of data blocks. Each handle has an owner that is allowed to read and write data to it as well as transfer its ownership to another user.

We are also assuming a crash-resistant transactional storage system where every API call is executed atomically. If a crash happens during an API call, then effects of the call either happens completely or doesn't happen at all.

We assume system is running on a disk with asynchronious buffered writes. Until an explicit sync operation is performeds, it is not guaranteed writes to be persisted on the disk. Ones that are persisted also could be out of order.

\subsection*{Running example}
Throughout the paper, we will follow the scenario where a user writes some data to a file and then transfers ownership of the file to another user.
\begin{lstlisting}
append_and_transfer(handle, data, new_owner)
  extend(handle, data)
  change_owner(handle, new_owner)
\end{lstlisting}
Example is simple enough to not have any distracting details but complex enough to reveal subtleties involved in a possibly malicious storage system implementation.\\

Possible problems that can arise in implementation of write:

- System can stash away the input confidential data. 
(Same as reading the file and stashing it somewhere?)

- System can write data to someone else's file

- System can leak data due to its crash-resistance mechanisms.\\

Possible problems that can arise in implementation of change\_owner:

- System can give ownership to someone else,

- System can write some confidential data belonging to the current user 
to the file before transferring it to new owner.

- System can read file's content then reveal it some other time.\\

If we generalize these problems we can have following general groups:

- Hiding confidential data in a place that gets abstracted away.
This place will be referred as "the stash" and the action of hiding data there as "stashing". This data either can be the new data that is provided to the system as an input to an API call or already existing data that is read by the system during an API call.

- Transferring an existing data to another user. This could be the stashed or already on-disk data.

- Revealing data contents via different crash-and-recovery outcomes.

\end{comment}


%Just like functionality of systems can be precisely defined in terms of specifications, desired properties of those systems can be precisely defined in terms of system properties. A class of these properties that encompasses security and liveness is called hyperproperties. Unlike properties that depends on single run of a system, hyperproperties relate multiple runs of a system. Because of this distinction, hyperproperties of a system are harder to prove and are not preserved under simulation based refinement. A strong notion of simulation called bisimulation is required to preserve hyperproperties. However bisimulation requirement reduces the power of abstraction significantly.

%This poses a challenge to the modular approaches to proving hyperproperties. Since simulation is a core technique to modularly proving an implementation behaving similar to an abstract representation.\\

%//Example here\\



